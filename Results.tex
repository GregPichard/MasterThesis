\section{Panel regression results and discussion}\footnote{The source code of all data queries, treatments and analysis output can be found online at \url{https://github.com/GregPichard/MasterThesis.git}. The repository is private, at the time of the writing of this thesis, and access can granted for clone or download upon request. Nonetheless, the necessary data cannot be stored there due to the important disk space required.}
\label{sec:Results}
In this whole section, descriptive statistics and results will always be given in the same order : first the U.S. stocks' sample, which is a tentative replication of \textcite{Ben-David2018} and then the international (non-U.S.) stocks' sample, which is one of the newer contribution of this paper. The model specifications tested are the same across both samples, which makes qualitative comparisons between both distinct populations more intuitive. All regressions are performed using a heteroskedasticity and auto-correlation robust (HAC) covariance estimator following the methodology introduced in \textcite{Driscoll1998} for panel and spatial regressions based on the widely-used HAC covariance estimator from \textcite{Newey1987}. The mathematical detail of this covariance matrix can be found in \autoref{app:sec:EconometricDefinitions}, p.\pageref{app:sec:EconometricDefinitions}.

Through comparisons that are not displayed here, it has been steadily noticed that the \textcite{Newey1987} tends to yield more conservative estimates (thus lower t-stats in absolute value) than the heteroskedasticity-robust estimator with clustering both on entities (stocks) and time. Nevertheless, and that is why such comparisons are not disclosed, the degree of significance of coefficient estimates does not change the conclusions drawn from the three models tested.  The estimation methodology, whose implementation is automated through a statistical package, is explained in \autoref{app:sec:EconometricDefinitions} as well as several statistics displayed in the exhibits.

\subsection{Summary statistics}
This section aims first at providing an overview about the distribution of all variables in the upcoming regressions. In addition, although a thorough comparison with the emerging literature about ETF ownership effects on securities is beyond the scope of this paper, summary statistics help ensuring that the sample bears some resemblance with those used in other studies. Resemblance is obviously weaker, less precise and rigorous than similarity and one should for instance perform statistical tests such as the F-test or Bartlett's test for equal variances across populations, under normality assumptions, if two samples are available.

Last, summary statistics are also useful in order to filter outliers at both ends of distributions. The general policy for this matter is first to exclude data points that are obviously wrong because they are impossible : a typical situation of a ratio $\frac{\mathsf{\#ETF-owned\_shares}}{\mathsf{\#Shares\_outstanding}}$ above 100\% is excluded but some cases are critical because doubt is allowed : a delay in the reporting of fund holdings can yield ETF ownership ratios close to, and yet below 1. Such case, however unlikely, may as well happen and no assumption has been made regarding a maximum tolerated ETF ownership share because of a lack of relevant information. Then, for the sake of simplicity, variables are truncated rather than winsorised at the 99.99\% quantile; the choice of such a high threshold has been guided by the fact that, in some variables including the essential ETF ownership share, variance concentrates in the upper percentiles of the distribution. In other terms, this variable empirical distribution is right-skewed and the intuition is that higher percentiles are not necessarily irrelevant outliers, they may even constitute an important part of our sample. For instance, the international sample exhibits very low share of (U.S.-listed) ETF ownership: the median and the 75th percentiles are 0, the 90th is equal to 0.7\%, the 95th to 1.6\% and the 99th to 4.3\%. The value exceeded by only 0.01\% of the sample, and therefore considered the threshold for truncation, is 22.5\%. Although not disclosed in summary tables, ownership of other institutional investors are especially has an especially outside the U.S., which in turn raises two issues in regressions : first the usable panel is a small fraction of the overall collected sample and even for the usable intersection, the rank condition on the regressors matrix is sometimes violated unless two of those controls, the pension and hedge funds share of ownership, are excluded from the regression.

The summary statistics in \autoref{tab:SummaryTable:US} and \autoref{tab:SummaryTable:International}, for the U.S. and abroad respectively, are computed entire samples, hence before truncation; only unvalaible values, encoded as \texttt{NaN} (``not a number'') or as infinite, are excluded. The maxima are therefore absent from data used in following regressions, which seems a healthy precaution for ETF ownership as explained above, but also for several controls including notably the book-to-market ratio, the percent bid-ask spread and the previous 12-month return.

\autoref{subtab:SummaryTable:US:PanelB} and \autoref{subtab:SummaryTable:International:PanelB} are essentially used to control for low absolute linear correlation between regressors in order to avoid any risk of collinearity among them; such risk seems absent and no coefficient is above 0.5 in absolute value. Perhaps anecdotally, the ``naïve'' univariate correlation between ETF ownership and the volatility of daily stock returns is negative (approx. $-0.23$) in the U.S. and very close to zero outside the U.S. Similarly, the bid-ask spread, a measure of liquidity which is assumed inversely linked with liquidity, exhibits a negative $\rho$ with ETF ownership while we may expect, in our multivariate tests, a positive correlation.
{\linespread{1.0}
\begin{landscape}
\begin{table}[htbp]
\caption{U.S. Sample (monthly) : Summary Statistics}
\label{tab:SummaryTable:US}
\begin{subtable}[t]{\linewidth}
\subcaption{Summary statistics}
\label{subtab:SummaryTable:US:PanelA}
\begin{tabular}{lrrrrrrrr}
\toprule
{} & N (obs.) &    Mean & St. dev. &      Min. &    25\% &  Median &     75\% &       Max. \\
\midrule
Volatility                &   296405 &   0.026 &    0.020 &     0.001 &  0.015 &   0.021 &   0.031 &      1.618 \\
ETF Ownership             &   296405 &   0.028 &    0.035 &     0.000 &  0.001 &   0.016 &   0.043 &      0.990 \\
Book-to-market            &   296405 &   1.008 &  157.773 &  -196.776 &  0.243 &   0.440 &   0.742 &  79321.012 \\
Market cap. (\$ Mln.)      &   296405 & 699.951 & 2833.749 &     0.013 & 29.986 & 105.878 & 365.001 & 109943.606 \\
1/Price                   &   296405 &   0.089 &    0.120 &     0.000 &  0.026 &   0.048 &   0.097 &      1.000 \\
Rel. Bid-Ask spread       &   296405 &   0.005 &    0.014 &    -0.015 &  0.001 &   0.001 &   0.004 &      1.765 \\
Amihud ratio              &   296405 &   0.000 &    0.000 &     0.000 &  0.000 &   0.000 &   0.000 &      0.003 \\
Past 12-to-1-month return &   296405 &   0.155 &    0.654 &    -1.000 & -0.144 &   0.075 &   0.320 &     43.375 \\
Past 12-to-7-month return &   296405 &  -0.049 &    3.632 & -1868.259 & -0.120 &   0.037 &   0.161 &      0.965 \\
Gross profitability       &   296405 &   0.333 &    0.351 &    -0.850 &  0.149 &   0.272 &   0.439 &     36.028 \\
\bottomrule
\end{tabular}

\end{subtable}
%\newline
\medskip
\newline
\begin{subtable}[t]{\linewidth}
\subcaption{Pearson correlation coefficients}
\label{subtab:SummaryTable:US:PanelB}
\begin{tabular}{llrrrrrrrrrr}
\toprule
                    &      &    (1) &    (2) &    (3) &    (4) &    (5) &    (6) &    (7) &   (8) &   (9) &  (10) \\
\midrule
Volatility & (1) &  1.000 &        &        &        &        &        &        &       &       &       \\
ETF Ownership & (2) & -0.149 &  1.000 &        &        &        &        &        &       &       &       \\
Book-to-market & (3) &  0.005 & -0.003 &  1.000 &        &        &        &        &       &       &       \\
Market cap. (\$ Mln.) & (4) & -0.101 &  0.028 & -0.001 &  1.000 &        &        &        &       &       &       \\
1/Price & (5) &  0.286 & -0.217 &  0.004 & -0.126 &  1.000 &        &        &       &       &       \\
Rel. Bid-Ask spread & (6) &  0.251 & -0.186 &  0.000 & -0.072 &  0.312 &  1.000 &        &       &       &       \\
Amihud ratio & (7) &  0.085 & -0.025 & -0.001 & -0.009 &  0.083 &  0.114 &  1.000 &       &       &       \\
Past 12-to-1-month return & (8) & -0.052 & -0.001 & -0.005 &  0.002 & -0.089 & -0.065 & -0.018 & 1.000 &       &       \\
Past 12-to-7-month return & (9) & -0.035 &  0.003 & -0.005 &  0.005 & -0.026 & -0.013 & -0.004 & 0.010 & 1.000 &       \\
Gross profitability & (10) &  0.001 & -0.008 &  0.000 &  0.001 &  0.003 &  0.002 & -0.001 & 0.052 & 0.002 & 1.000 \\
\bottomrule
\end{tabular}

\end{subtable}
\end{table}
\clearpage
\begin{table}[htbp]
\caption{International Sample : Summary Statistics}
\label{tab:SummaryTable:International}
\begin{subtable}[t]{\linewidth}
\subcaption{Summary statistics}
\label{subtab:SummaryTable:International:PanelA}
\begin{tabular}{lrrrrrrrr}
\toprule
{} & N (obs.) &      Mean &   St. dev. &       Min. &     25\% &  Median &      75\% &         Max. \\
\midrule
Volatility                &  1338479 &     0.420 &      0.125 &      0.000 &   0.397 &   0.418 &    0.441 &      109.046 \\
ETF Ownership             &  1338479 &     0.003 &      0.012 &      0.000 &   0.000 &   0.000 &    0.001 &        0.950 \\
Book-to-market            &  1338479 &     0.875 &     33.362 & -30363.758 &   0.360 &   0.693 &    1.237 &     1102.476 \\
Market cap. (\$ Mln.)      &  1338479 & 14351.004 & 214744.406 &      0.001 & 113.098 & 609.315 & 3385.835 & 39864005.168 \\
1/Price                   &  1338479 &     0.082 &      0.551 &      0.000 &   0.001 &   0.004 &    0.060 &      333.333 \\
Rel. Bid-Ask spread       &  1338479 &     0.013 &      0.039 &     -2.000 &   0.002 &   0.005 &    0.013 &        2.000 \\
Amihud ratio              &  1338479 &     0.000 &      0.004 &     -0.000 &   0.000 &   0.000 &    0.000 &        2.450 \\
Past 12-to-1-month return &  1338479 &     0.162 &      1.381 &     -1.000 &  -0.159 &   0.048 &    0.317 &     1249.000 \\
Past 12-to-7-month return &  1338479 &    -0.036 &      6.559 &  -5282.798 &  -0.128 &   0.019 &    0.151 &        0.997 \\
Gross profitability       &  1338479 &     0.282 &      0.389 &    -79.835 &   0.121 &   0.215 &    0.365 &       76.409 \\
\bottomrule
\end{tabular}

\end{subtable}
\medskip
\newline
\begin{subtable}[t]{\linewidth}
\subcaption{Pearson correlation coefficients}
\label{subtab:SummaryTable:International:PanelB}
\begin{tabular}{llrrrrrrrrrr}
\toprule
                    &  & (1) & (2) &  (3) & (4) & (5) & (6) & (7) & (8) & (9) & (10)\\
\midrule
Volatility & (1) &  1.000 &        &        &        &        &        &        &       &        &       \\
ETF Ownership & (2) & -0.004 &  1.000 &        &        &        &        &        &       &        &       \\
Book-to-market & (3) & -0.001 & -0.002 &  1.000 &        &        &        &        &       &        &       \\
Market cap. (\$ Mln.) & (4) & -0.000 &  0.079 & -0.000 &  1.000 &        &        &        &       &        &       \\
1/Price & (5) &  0.000 & -0.029 & -0.002 & -0.011 &  1.000 &        &        &       &        &       \\
Rel. Bid-Ask spread & (6) &  0.018 & -0.073 & -0.002 & -0.018 &  0.175 &  1.000 &        &       &        &       \\
Amihud ratio & (7) &  0.000 & -0.001 & -0.000 & -0.000 &  0.021 &  0.044 &  1.000 &       &        &       \\
Past 12-to-1-month return & (8) & -0.000 & -0.003 &  0.000 &  0.001 & -0.004 &  0.003 & -0.000 & 1.000 &        &       \\
Past 12-to-7-month return & (9) &  0.000 &  0.000 &  0.000 &  0.000 &  0.000 & -0.011 & -0.000 & 0.001 &  1.000 &       \\
Gross profitability & (10) & -0.000 & -0.010 &  0.021 & -0.004 & -0.003 &  0.012 & -0.001 & 0.003 & -0.040 & 1.000 \\
\bottomrule
\end{tabular}

\end{subtable}
\end{table}
\end{landscape}
}


\subsection{ETF ownership and underlying stocks' volatility}
\label{sec:Results:sub:Volatility}
\subsubsection{U.S. sample}
\autoref{tab:Volatility:US:Comp} contains four different model specifications for which the coefficient on ETF ownership is always significant at a 1\% error level. From the first model (column \textit{Baseline}), one can interpret it as follows, on average, a 10 percentage point increase in the ETF interest in a stock is correlated with a 2.6 percentage point increase in this stock's daily returns over the same month. This baseline specification only uses controls relative to size, value and momentum. The column titled \textit{Controls + Volatility lags} tests the robustness using additional liquidity (\textcite{Amihud2002} ratio and bid-ask spread) and profitability, as in \textcite{Ben-David2018} and addresses the issue of serial correlation in the monthly volatility of returns using four lags of the dependent, all of them being positive and strongly significant. Subsequently, the magnitude of ETF ownership effect drops, from about $0.30$ to $0.04$. The third specification goes further in addressing the possible omitted variable bias due to institutional ownership : in order to distinguish the suspected correlation between ETF and other institutional flows into securities, three different categories of fund ownership are found to have an impact on volatility. Nevertheless, the signs are hard to justify at this stage : mutual funds' impact seems very close to zero, while an obvious contradiction\footnote{\dots despite checks performed on how the variables have been computed. We have to keep in mind the level of reporting for those categories may be far less precise than for mutual funds in the Thomson Reuters database, both in terms of representativeness and reporting frequency.} exists between seemingly stabilizing hedge funds andvolatility-impounding pension funds (!) Perhaps reassuringly, the effect of ETFs alone does not change substantially as a consequence, showing that the omitted variable bias did not matter; in other words, the relationship between ETFs' weight in the capital of company and its volatility is independent from other collective investment schemes. The standardization in the fourth and last column to the right means that the dependent and the main regressor of interest are centered, i.e. the sample mean is subtracted from the original variable, and divided by the sample standard deviation. The interpretation follows : a one standard deviation change in ETF ownership is related with a 0.58\% to 1.05\% (average : 0.81\%) of a standard deviation increase in the stock's volatility. In general, clear support can be found in favor of the liquidity trading hypothesis in the sample of stocks traded in the United States.

Regarding controls in the most complete and standardized model, we notice that size surprisingly appears as a negative factor, although economically very small; volatility loads positively on the book-to-market as well as liquidity-related factors (bid-ask spread and \textcite{Amihud2002} ratio). Profitability is negative at 1\% level but economically near zero and momentum is negative at 5\% and again causes little moves. At 10 percentage point increase in the recent return would apparently reduce volatility by 2 basis points. As for all models in this paper, the coefficient of determination of all estimated models (``standard''$ R^2$) is far lower than values displayed in analogous published studies that have served as guidance. Without any access to their sample and estimation methods further than minimal disclosures, it is not possible to come up with any rationale.

%Although statistically different from no effect, this result has arguably little economic magnitude: almost no pairwise correlation between these two variables can be found in general. This sample contains an important proportion of small stocks: the average market capitalization is worth USD 717 million while the median is USD 107; one quarter of the sample is populated with observations below USD 30 million. 


\begin{landscape}
  {\linespread{1.0}
    %\begin{table}[htbp]
    % \begin{center}
  \begin{tabular}{lcccc}
    \caption{U.S. Sample : Exchange-Traded Fund aggregate ownership share and the volatility of underlying securities' daily returns}
    \label{tab:Volatility:US:Comp}

  \toprule
                                           & \textbf{Baseline}  & \textbf{Controls + Vol. lags} & \textbf{Inst. o'ship controls} & \textbf{Standardized}  \\
\midrule
\textbf{Dep. Variable}                     &     Volatility     &           Volatility          &           Volatility           &       Volatility       \\
\textbf{Estimator}                         &      PanelOLS      &            PanelOLS           &            PanelOLS            &        PanelOLS        \\
\textbf{No. Observations}                  &       413304       &             297405            &             297399             &         297247         \\
\textbf{Cov. Est.}                         &   Driscoll-Kraay   &         Driscoll-Kraay        &         Driscoll-Kraay         &     Driscoll-Kraay     \\
\textbf{R-squared}                         &       0.0545       &             0.1592            &             0.1593             &         0.1594         \\
\textbf{R-Squared (Within)}                &       0.0565       &             0.1353            &             0.1351             &         0.1348         \\
\textbf{R-Squared (Between)}               &       0.2009       &             0.7626            &             0.7625             &         0.7616         \\
\textbf{R-Squared (Overall)}               &       0.1515       &             0.2716            &             0.2715             &         0.2710         \\
\textbf{F-statistic}                       &       4717.9       &             4643.9            &             3718.4             &         3718.6         \\
\textbf{P-value (F-stat)}                  &       0.0000       &             0.0000            &             0.0000             &         0.0000         \\
\midrule
\textbf{Intercept}                         &       0.2964       &             0.0488            &             0.0494             &        -0.0154         \\
\textbf{ }                                 &      (22.268)      &            (7.1942)           &            (7.2977)            &       (-0.4273)        \\
\textbf{PctSharesHeldETF\_1lag}            &       0.2470       &             0.0385            &             0.0395             &         0.0081         \\
\textbf{ }                                 &      (8.2542)      &            (5.9800)           &            (6.2940)            &        (6.8269)        \\
\textbf{np.log(CompanyMarketCap\_1lag)}    &      -0.0127       &            -0.0018            &            -0.0018             &        -0.0102         \\
\textbf{ }                                 &     (-20.751)      &           (-5.8479)           &           (-5.9701)            &       (-6.1991)        \\
\textbf{InvClose\_1lag}                    &       0.0988       &             0.0251            &             0.0251             &         0.1452         \\
\textbf{ }                                 &      (10.972)      &            (7.7261)           &            (7.7792)            &        (7.9151)        \\
\textbf{BookToMarketRatio\_1lag}           &     5.552e-07      &           7.393e-07           &           1.038e-06            &       5.019e-06        \\
\textbf{ }                                 &      (2.3023)      &            (4.1534)           &            (8.5166)            &        (5.3019)        \\
\textbf{RetPast12to1M\_1lag}               &      -0.0003       &            -0.0004            &            -0.0004             &        -0.0024         \\
\textbf{ }                                 &     (-0.6844)      &           (-1.8758)           &           (-1.8564)            &       (-1.8927)        \\
\textbf{AmihudRatio\_1lag}                 &                    &             3.4590            &             3.4611             &         19.952         \\
\textbf{ }                                 &                    &            (2.7034)           &            (2.7029)            &        (2.7038)        \\
\textbf{PctBidAskSpread\_1lag}             &                    &             0.1750            &             0.1748             &         1.0060         \\
\textbf{ }                                 &                    &            (4.5505)           &            (4.5446)            &        (4.5471)        \\
\textbf{GrossProfitability\_1lag}          &                    &            -0.0005            &            -0.0005             &        -0.0028         \\
\textbf{ }                                 &                    &           (-2.6218)           &           (-2.6334)            &       (-2.6197)        \\
\textbf{Volatility\_1lag}                  &                    &             0.1377            &             0.1378             &         0.1376         \\
\textbf{ }                                 &                    &            (6.7584)           &            (6.7701)            &        (6.7630)        \\
\textbf{Volatility\_2lag}                  &                    &             0.1605            &             0.1604             &         0.1603         \\
\textbf{ }                                 &                    &            (16.002)           &            (15.972)            &        (15.949)        \\
\textbf{Volatility\_3lag}                  &                    &             0.1230            &             0.1229             &         0.1229         \\
\textbf{ }                                 &                    &            (13.390)           &            (13.402)            &        (13.403)        \\
\textbf{Volatility\_4lag}                  &                    &             0.0819            &             0.0818             &         0.0817         \\
\textbf{ }                                 &                    &            (9.8851)           &            (9.8688)            &        (9.8756)        \\
\textbf{PctSharesHeldOtherMutual\_1lag}    &                    &                               &            -0.0003             &        -0.0016         \\
\textbf{ }                                 &                    &                               &           (-2.5676)            &       (-1.1497)        \\
\textbf{PctSharesHeldPension\_1lag}        &                    &                               &             1.0330             &         0.0015         \\
\textbf{ }                                 &                    &                               &            (3.4034)            &        (3.6451)        \\
\textbf{PctSharesHeldHedge\_1lag}          &                    &                               &            -0.0284             &       9.023e-05        \\
\textbf{ }                                 &                    &                               &           (-3.2527)            &        (0.4613)        \\
\midrule
\textbf{Effects}                           &       Entity       &             Entity            &             Entity             &         Entity         \\
\bottomrule
\end{tabular}
%\caption{Model Comparison}
\end{center}
T-stats reported in parentheses.

    %\end{table}
    {\small
\begin{center}
  \begin{longtable}{lcccc}
\linespread{1.0}\\
\multicolumn{5}{r}{\textit{Continued from previous page}}\\
\toprule
     & \textbf{Baseline}  & \textbf{Controls + Vol. lags} & \textbf{Inst. o'ship controls} & \textbf{Standardized}  \\
\midrule
\endhead
\caption{U.S. Sample : Exchange-Traded Fund aggregate ownership share and the volatility of underlying securities' daily returns}\\
\label{tab:Volatility:US:Comp}\\
\toprule
     & \textbf{Baseline}  & \textbf{Controls + Vol. lags} & \textbf{Inst. o'ship controls} & \textbf{Standardized}  \\
\midrule
\endfirsthead
\bottomrule
\multicolumn{5}{r}{\textit{Continues next page}}\\
\endfoot
\bottomrule
\endlastfoot
\textbf{Dep. Variable}                     &     Volatility     &           Volatility          &           Volatility           &       Volatility       \\
\textbf{Estimator}                         &      PanelOLS      &            PanelOLS           &            PanelOLS            &        PanelOLS        \\
\textbf{No. Observations}                  &       413304       &             297405            &             297399             &         297247         \\
\textbf{Cov. Est.}                         &   Driscoll-Kraay   &         Driscoll-Kraay        &         Driscoll-Kraay         &     Driscoll-Kraay     \\
\textbf{R-squared}                         &       0.0545       &             0.1592            &             0.1593             &         0.1594         \\
\textbf{R-Squared (Within)}                &       0.0565       &             0.1353            &             0.1351             &         0.1348         \\
\textbf{R-Squared (Between)}               &       0.2009       &             0.7626            &             0.7625             &         0.7616         \\
\textbf{R-Squared (Overall)}               &       0.1515       &             0.2716            &             0.2715             &         0.2710         \\
\textbf{F-statistic}                       &       4717.9       &             4643.9            &             3718.4             &         3718.6         \\
\textbf{P-value (F-stat)}                  &       0.0000       &             0.0000            &             0.0000             &         0.0000         \\
\midrule
\textbf{Intercept}                         &       0.2964       &             0.0488            &             0.0494             &        -0.0154         \\
\textbf{ }                                 &      (22.268)      &            (7.1942)           &            (7.2977)            &       (-0.4273)        \\
\textbf{PctSharesHeldETF\_1lag}            &       0.2470       &             0.0385            &             0.0395             &         0.0081         \\
\textbf{ }                                 &      (8.2542)      &            (5.9800)           &            (6.2940)            &        (6.8269)        \\
\textbf{np.log(CompanyMarketCap\_1lag)}    &      -0.0127       &            -0.0018            &            -0.0018             &        -0.0102         \\
\textbf{ }                                 &     (-20.751)      &           (-5.8479)           &           (-5.9701)            &       (-6.1991)        \\
\textbf{InvClose\_1lag}                    &       0.0988       &             0.0251            &             0.0251             &         0.1452         \\
\textbf{ }                                 &      (10.972)      &            (7.7261)           &            (7.7792)            &        (7.9151)        \\
\textbf{BookToMarketRatio\_1lag}           &     5.552e-07      &           7.393e-07           &           1.038e-06            &       5.019e-06        \\
\textbf{ }                                 &      (2.3023)      &            (4.1534)           &            (8.5166)            &        (5.3019)        \\
\textbf{RetPast12to1M\_1lag}               &      -0.0003       &            -0.0004            &            -0.0004             &        -0.0024         \\
\textbf{ }                                 &     (-0.6844)      &           (-1.8758)           &           (-1.8564)            &       (-1.8927)        \\
\textbf{AmihudRatio\_1lag}                 &                    &             3.4590            &             3.4611             &         19.952         \\
\textbf{ }                                 &                    &            (2.7034)           &            (2.7029)            &        (2.7038)        \\
\textbf{PctBidAskSpread\_1lag}             &                    &             0.1750            &             0.1748             &         1.0060         \\
\textbf{ }                                 &                    &            (4.5505)           &            (4.5446)            &        (4.5471)        \\
\textbf{GrossProfitability\_1lag}          &                    &            -0.0005            &            -0.0005             &        -0.0028         \\
\textbf{ }                                 &                    &           (-2.6218)           &           (-2.6334)            &       (-2.6197)        \\
\textbf{Volatility\_1lag}                  &                    &             0.1377            &             0.1378             &         0.1376         \\
\textbf{ }                                 &                    &            (6.7584)           &            (6.7701)            &        (6.7630)        \\
\pagebreak
\textbf{Volatility\_2lag}                  &                    &             0.1605            &             0.1604             &         0.1603         \\
\textbf{ }                                 &                    &            (16.002)           &            (15.972)            &        (15.949)        \\
\textbf{Volatility\_3lag}                  &                    &             0.1230            &             0.1229             &         0.1229         \\
\textbf{ }                                 &                    &            (13.390)           &            (13.402)            &        (13.403)        \\
\textbf{Volatility\_4lag}                  &                    &             0.0819            &             0.0818             &         0.0817         \\
\textbf{ }                                 &                    &            (9.8851)           &            (9.8688)            &        (9.8756)        \\
\textbf{PctSharesHeldOtherMutual\_1lag}    &                    &                               &            -0.0003             &        -0.0016         \\
\textbf{ }                                 &                    &                               &           (-2.5676)            &       (-1.1497)        \\
\textbf{PctSharesHeldPension\_1lag}        &                    &                               &             1.0330             &         0.0015         \\
\textbf{ }                                 &                    &                               &            (3.4034)            &        (3.6451)        \\
\textbf{PctSharesHeldHedge\_1lag}          &                    &                               &            -0.0284             &       9.023e-05        \\
\textbf{ }                                 &                    &                               &           (-3.2527)            &        (0.4613)        \\
\midrule
\textbf{Effects}                           &       Entity       &             Entity            &             Entity             &         Entity         \\
& Time &  Time  &  Time &  Time\\
  \end{longtable}

\end{center}
T-stats reported in parentheses.
}

  }
\end{landscape}

\subsubsection{International sample}
In the non-U.S. sample (\autoref{tab:Volatility:Intl:Comp}), coefficients generally bear less statistical significance than in a their American counterpart. ETF ownership is only 10\%-significant (positive impact) until liquidity controls and volatility lags are introduced, hence one cannot conclude, based on this evidence, that the share of a stock's capital held through ETFs plays any role on this stock's next-month volatility. It is also worth noting that almost all the variance of observed volatility explained in thos models, nearly 16\% in the U.S. and 17\% outside, is due to four volatility lags, whose coefficients exhibit a strong positive autocorrelation. Significant ($\alpha = 1\%$) controls are the market capitalization (with a negative sign), the book-to-market ratio (value stocks are more volatile) and the relative bid-ask spread, acting as an illiquidity proxy.

\bigskip
For the sake of completeness, more detail is reported in the appendix, \autoref{app:sec:DetailedResults:Volatility}, p.\pageref{app:sec:DetailedResults:Volatility}, about the panel estimation results testing the relationship between the share of ETF ownership in securities and the held securities' daily returns volatility. 

\begin{landscape}
  {\linespread{1.0}
    %\begin{table}[htbp]
    % \begin{center}
\begin{tabular}{lcccc}
\toprule
                                           & \textbf{Baseline}  & \textbf{Controls + Vol. lags} & \textbf{Inst. o'ship control} & \textbf{Standardized}  \\
\midrule
\textbf{Dep. Variable}                     &     Volatility     &           Volatility          &           Volatility          &       Volatility       \\
\textbf{Estimator}                         &      PanelOLS      &            PanelOLS           &            PanelOLS           &        PanelOLS        \\
\textbf{No. Observations}                  &      1516791       &            1319966            &            1319966            &        1319966         \\
\textbf{Cov. Est.}                         &   Driscoll-Kraay   &         Driscoll-Kraay        &         Driscoll-Kraay        &     Driscoll-Kraay     \\
\textbf{R-squared}                         &       0.0070       &             0.1729            &             0.1729            &         0.1729         \\
\textbf{R-Squared (Within)}                &       0.0148       &             0.2100            &             0.2100            &         0.2100         \\
\textbf{R-Squared (Between)}               &      -0.0101       &             0.7081            &             0.7079            &         0.7079         \\
\textbf{R-Squared (Overall)}               &      -0.0208       &             0.3261            &             0.3260            &         0.3260         \\
\textbf{F-statistic}                       &       2118.7       &           2.274e+04           &           2.099e+04           &       2.099e+04        \\
\textbf{P-value (F-stat)}                  &       0.0000       &             0.0000            &             0.0000            &         0.0000         \\
\textbf{===============================}   &  ================  &        ================       &        ================       &    ================    \\
\textbf{Intercept}                         &       0.0582       &             0.0283            &             0.0283            &         0.6235         \\
\textbf{ }                                 &      (10.611)      &            (7.7563)           &            (7.7708)           &        (4.4628)        \\
\textbf{PctSharesHeldETF\_1lag}            &       0.0169       &             0.0072            &             0.0070            &         0.0028         \\
\textbf{ }                                 &      (1.8226)      &            (1.0225)           &            (0.9781)           &        (0.9781)        \\
\textbf{np.log(CompanyMarketCap\_1lag)}    &      -0.0015       &            -0.0008            &            -0.0008            &        -0.0310         \\
\textbf{ }                                 &     (-6.1428)      &           (-4.9380)           &           (-4.9511)           &       (-4.9511)        \\
\textbf{InvClose\_1lag}                    &       0.0045       &             0.0015            &             0.0015            &         0.0585         \\
\textbf{ }                                 &      (2.5792)      &            (1.1485)           &            (1.1484)           &        (1.1484)        \\
\textbf{BookToMarketRatio\_1lag}           &     3.446e-05      &            2.54e-05           &           2.538e-05           &         0.0010         \\
\textbf{ }                                 &      (3.6160)      &            (3.3179)           &            (3.3206)           &        (3.3206)        \\
\textbf{RetPast12to1M\_1lag}               &       0.0005       &             0.0001            &             0.0001            &         0.0052         \\
\textbf{ }                                 &      (1.8965)      &            (1.5202)           &            (1.5203)           &        (1.5203)        \\
\textbf{AmihudRatio\_1lag}                 &                    &             0.0250            &             0.0250            &         0.9984         \\
\textbf{ }                                 &                    &            (0.8272)           &            (0.8272)           &        (0.8272)        \\
\textbf{PctBidAskSpread\_1lag}             &                    &             0.0262            &             0.0262            &         1.0496         \\
\textbf{ }                                 &                    &            (6.9533)           &            (6.9532)           &        (6.9532)        \\
\textbf{GrossProfitability\_1lag}          &                    &            -0.0001            &            -0.0001            &        -0.0056         \\
\textbf{ }                                 &                    &           (-0.6729)           &           (-0.6731)           &       (-0.6731)        \\
\textbf{Volatility\_1lag}                  &                    &             0.2541            &             0.2541            &         0.2541         \\
\textbf{ }                                 &                    &            (21.616)           &            (21.615)           &        (21.615)        \\
\textbf{Volatility\_2lag}                  &                    &             0.1226            &             0.1226            &         0.1226         \\
\textbf{ }                                 &                    &            (18.960)           &            (18.960)           &        (18.960)        \\
\textbf{Volatility\_3lag}                  &                    &             0.0968            &             0.0968            &         0.0968         \\
\textbf{ }                                 &                    &            (11.790)           &            (11.790)           &        (11.790)        \\
\textbf{Volatility\_4lag}                  &                    &             0.0607            &             0.0607            &         0.0607         \\
\textbf{ }                                 &                    &            (13.274)           &            (13.271)           &        (13.271)        \\
\textbf{PctSharesHeldOtherMutual\_1lag}    &                    &                               &           9.061e-05           &         0.0008         \\
\textbf{ }                                 &                    &                               &            (0.6096)           &        (0.6096)        \\
\textbf{=================================} & ================== &       ==================      &       ==================      &   ==================   \\
\textbf{Effects}                           &       Entity       &             Entity            &             Entity            &         Entity         \\
\bottomrule
\end{tabular}
%\caption{Model Comparison}
\end{center}

T-stats reported in parentheses
    %\end{table}
    {\small
  \begin{center}
    \begin{longtable}{lcccc}
      \linespread{1.0}\\
      \multicolumn{5}{r}{\textit{Continued from previous page}}\\
      \toprule
      & \textbf{Baseline}  & \textbf{Controls + Vol. lags} & \textbf{Inst. o'ship control} & \textbf{Standardized}  \\
      \midrule
      \endhead
      \caption{International Sample : Exchange-Traded Fund aggregate ownership share and the volatility of underlying securities' daily returns}\\
      \label{tab:Volatility:Intl:Comp}\\
      \toprule
      & \textbf{Baseline}  & \textbf{Controls + Vol. lags} & \textbf{Inst. o'ship control} & \textbf{Standardized}  \\
      \midrule
      \endfirsthead
      \bottomrule
      \multicolumn{5}{r}{\textit{Continues on next page}}\\
      \endfoot
      \bottomrule
      \endlastfoot
      \textbf{Dep. Variable}                     &     Volatility     &           Volatility          &           Volatility          &       Volatility       \\
      \textbf{Estimator}                         &      PanelOLS      &            PanelOLS           &            PanelOLS           &        PanelOLS        \\
      \textbf{No. Observations}                  &      1516791       &            1319966            &            1319966            &        1319966         \\
      \textbf{Cov. Est.}                         &   Driscoll-Kraay   &         Driscoll-Kraay        &         Driscoll-Kraay        &     Driscoll-Kraay     \\
      \textbf{R-squared}                         &       0.0070       &             0.1729            &             0.1729            &         0.1729         \\
      \textbf{R-Squared (Within)}                &       0.0148       &             0.2100            &             0.2100            &         0.2100         \\
      \textbf{R-Squared (Between)}               &      -0.0101       &             0.7081            &             0.7079            &         0.7079         \\
      \textbf{R-Squared (Overall)}               &      -0.0208       &             0.3261            &             0.3260            &         0.3260         \\
      \textbf{F-statistic}                       &       2118.7       &           2.274e+04           &           2.099e+04           &       2.099e+04        \\
      \textbf{P-value (F-stat)}                  &       0.0000       &             0.0000            &             0.0000            &         0.0000         \\
      \midrule
      \textbf{Intercept}                         &       0.0582       &             0.0283            &             0.0283            &         0.6235         \\
      \textbf{ }                                 &      (10.611)      &            (7.7563)           &            (7.7708)           &        (4.4628)        \\
      \textbf{PctSharesHeldETF\_1lag}            &       0.0169       &             0.0072            &             0.0070            &         0.0028         \\
      \textbf{ }                                 &      (1.8226)      &            (1.0225)           &            (0.9781)           &        (0.9781)        \\
      \textbf{np.log(CompanyMarketCap\_1lag)}    &      -0.0015       &            -0.0008            &            -0.0008            &        -0.0310         \\
      \textbf{ }                                 &     (-6.1428)      &           (-4.9380)           &           (-4.9511)           &       (-4.9511)        \\
      \textbf{InvClose\_1lag}                    &       0.0045       &             0.0015            &             0.0015            &         0.0585         \\
      \textbf{ }                                 &      (2.5792)      &            (1.1485)           &            (1.1484)           &        (1.1484)        \\
      \textbf{BookToMarketRatio\_1lag}           &     3.446e-05      &            2.54e-05           &           2.538e-05           &         0.0010         \\
      \textbf{ }                                 &      (3.6160)      &            (3.3179)           &            (3.3206)           &        (3.3206)        \\
      \textbf{RetPast12to1M\_1lag}               &       0.0005       &             0.0001            &             0.0001            &         0.0052         \\
      \textbf{ }                                 &      (1.8965)      &            (1.5202)           &            (1.5203)           &        (1.5203)        \\
      \textbf{AmihudRatio\_1lag}                 &                    &             0.0250            &             0.0250            &         0.9984         \\
      \textbf{ }                                 &                    &            (0.8272)           &            (0.8272)           &        (0.8272)        \\
      \textbf{PctBidAskSpread\_1lag}             &                    &             0.0262            &             0.0262            &         1.0496         \\
      \textbf{ }                                 &                    &            (6.9533)           &            (6.9532)           &        (6.9532)        \\
      \textbf{GrossProfitability\_1lag}          &                    &            -0.0001            &            -0.0001            &        -0.0056         \\
      \textbf{ }                                 &                    &           (-0.6729)           &           (-0.6731)           &       (-0.6731)        \\
      \textbf{Volatility\_1lag}                  &                    &             0.2541            &             0.2541            &         0.2541         \\
      \textbf{ }                                 &                    &            (21.616)           &            (21.615)           &        (21.615)        \\
      \textbf{Volatility\_2lag}                  &                    &             0.1226            &             0.1226            &         0.1226         \\
      \textbf{ }                                 &                    &            (18.960)           &            (18.960)           &        (18.960)        \\
      \textbf{Volatility\_3lag}                  &                    &             0.0968            &             0.0968            &         0.0968         \\
      \textbf{ }                                 &                    &            (11.790)           &            (11.790)           &        (11.790)        \\
      \textbf{Volatility\_4lag}                  &                    &             0.0607            &             0.0607            &         0.0607         \\
      \textbf{ }                                 &                    &            (13.274)           &            (13.271)           &        (13.271)        \\
      \textbf{PctSharesHeldOtherMutual\_1lag}    &                    &                               &           9.061e-05           &         0.0008         \\
      \textbf{ }                                 &                    &                               &            (0.6096)           &        (0.6096)        \\
      \midrule
      \textbf{Effects}                           &       Entity       &             Entity            &             Entity            &         Entity         \\
      & Time & Time & Time & Time\\
    \end{longtable}

  \end{center}
  T-stats reported in parentheses.
}

  }
\end{landscape}
\subsection{ETF ownership and underlying stocks' liquidity}
\label{sec:Results:sub:Liquidity}
\subsubsection{U.S. sample}
In \autoref{tab:Liquidity:US:Comp}, the analysis of ETF ownership on two liquidity proxies, the relative bid-ask spread and the \textcite{Amihud2002} ratio is displayed. For each dependent variable, two variants, respectively with and without institutional ownership controls, yield very mixed results regarding the negative influence of ETF ownership on liquidity : indeed, both the illiquidity ratio equations exhibit positive coefficients with very high t-statistics, and those effects do not originate from a possible correlation between various categories of funds (ETF, other mutual, hedge funds). These results lead us to think that the increased degree of ETF involvement in a company's equity on average causes transactions to become more costly and, importantly for other institutional investors, to impound larger absolute moves on prices. In the Amihud ratio numerator model (first and second columns), the coefficient on ETF ownership means that a one percentage point increase in this variable is associated with a 0.19 percentage point (or 19 basis points) increase in the daily average absolute return, its dollar volume of transactions and other specified controls being kept constant. Contrary to what \textcite{Israeli2017} exhibits regarding their bid-ask spread proxy, the coefficients on ETF ownership (as well as pension and hedge funds ownership) are not statistically significant from zero, and the negative one on other open-end mutual funds ownership (also noticed in the Amihud ratio model)  implies a very weak sensitivity. In those models, including entity and fixed effects strongly impair the $R^2$: a factor of 3 to 4 lies between the actual model $R^2$ and the \textit{overall} version (without fixed effects). The pooled OLS model was not considered because it is not clear whether common (global market conditions) or time-invariant (individual/industry-specific) characteristics, do not actually play a role in the spread and the price impact of trades.

\begin{landscape}
  {\linespread{1.0}
    \begin{table}[htbp]
      \begin{center}
  \caption{Exchange-Traded Funds' aggregate ownership share and underlying securities' liquidity}
  \label{tab:Liquidity:Comp}
\begin{tabular}{lcccc}
\toprule
                                           &   \textbf{Amihud}   & \textbf{Amihud w/inst. o'ship} &   \textbf{Bid-Ask}  & \textbf{Bid-Ask w/inst. o'ship}  \\
\midrule
\textbf{Dep. Variable}                     &   AmihudNumerator   &        AmihudNumerator         &   PctBidAskSpread   &         PctBidAskSpread          \\
\textbf{Estimator}                         &       PanelOLS      &            PanelOLS            &       PanelOLS      &             PanelOLS             \\
\textbf{No. Observations}                  &        436857       &             436852             &        335350       &              335346              \\
\textbf{Cov. Est.}                         &    Driscoll-Kraay   &         Driscoll-Kraay         &    Driscoll-Kraay   &          Driscoll-Kraay          \\
\textbf{R-squared}                         &        0.0330       &             0.0330             &        0.0512       &              0.0512              \\
\textbf{R-Squared (Within)}                &        0.0436       &             0.0436             &        0.0661       &              0.0663              \\
\textbf{R-Squared (Between)}               &        0.1873       &             0.1873             &        0.5612       &              0.5618              \\
\textbf{R-Squared (Overall)}               &        0.1276       &             0.1276             &        0.1882       &              0.1885              \\
\textbf{F-statistic}                       &        3688.0       &             2950.4             &        4465.5       &              2555.6              \\
\textbf{P-value (F-stat)}                  &        0.0000       &             0.0000             &        0.0000       &              0.0000              \\
\midrule
\textbf{Intercept}                         &        0.3438       &             0.3438             &        0.1054       &              0.1057              \\
\textbf{ }                                 &       (30.192)      &            (30.164)            &       (15.787)      &             (15.743)             \\
\textbf{PctSharesHeldETF}                  &        0.1870       &             0.1871             &        0.0061       &              0.0060              \\
\textbf{ }                                 &       (7.7614)      &            (7.7638)            &       (0.7923)      &             (0.7832)             \\
\textbf{np.log(CompanyMarketCap\_1lag)}    &       -0.0153       &            -0.0153             &       -0.0049       &             -0.0049              \\
\textbf{ }                                 &      (-26.847)      &           (-26.822)            &      (-14.639)      &            (-14.596)             \\
\textbf{BookToMarketRatio\_1lag}           &      2.894e-07      &           3.098e-07            &      -4.277e-08     &            -4.038e-08            \\
\textbf{ }                                 &       (1.1320)      &            (1.1432)            &      (-0.9008)      &            (-0.8852)             \\
\textbf{AmihudDenominator}                 &      1.321e-11      &            1.32e-11            &                     &                                  \\
\textbf{ }                                 &       (4.1972)      &            (4.1968)            &                     &                                  \\
\textbf{PctSharesHeldOtherMutual}          &                     &           -1.219e-05           &                     &            -2.114e-06            \\
\textbf{ }                                 &                     &           (-5.6177)            &                     &            (-4.5245)             \\
\textbf{Volatility\_1lag}                  &                     &                                &        0.0500       &              0.0499              \\
\textbf{ }                                 &                     &                                &       (10.104)      &             (10.107)             \\
\textbf{PctSharesHeldPension}              &                     &                                &                     &              0.3943              \\
\textbf{ }                                 &                     &                                &                     &             (1.2835)             \\
\textbf{PctSharesHeldHedge}                &                     &                                &                     &              0.0077              \\
\textbf{ }                                 &                     &                                &                     &             (1.7387)             \\
\midrule
\textbf{Effects}                           &        Entity       &             Entity             &        Entity       &              Entity              \\
& Time & Time & Time & Time\\
\bottomrule
\end{tabular}
\end{center}

T-stats reported in parentheses

    \end{table}
  }
\end{landscape}

\subsubsection{International sample}
From results summarized in \autoref{tab:Liquidity:Intl:Comp}, it is difficult to draw general conclusions regarding the impact of ETF interest in liquidity: here, and it is contrary to U.S. evidence above, no price impact, controlled with traded volume, can be found but a trading cost impact is reported with a high degree of likelihood ($\mathit{t-stat} = 4.95$). The \textcite{Amihud2002} denominator, to be found in the first two columns, had to be transformed into its logarithm in order to perform the regression, for a yet unknown reason. No role seems to be played by other available forms of fund investments, which have been reported in the literature to trade less frequently on average. Only the size factor (market capitalization) is always negative, reducing the price impact of trades (Amihud) and also convincing more dealers to compete for orders, thus lowering their margins (bid-ask spread).

\bigskip
Even though no universally significant effect has been found so far in this subsection, more detail is reported in the appendix, \autoref{app:sec:DetailedResults:Liquidity}, p.\pageref{app:sec:DetailedResults:Liquidity},  about the panel estimation results testing the relationship between the share of ETF ownership in securities and the statistics linked with stock-level liquidity (bid-ask spread, \textcite{Amihud2002} ratio decomposition). 

\begin{landscape}
  {\linespread{1.0}
    \begin{table}[htbp]
      \begin{center}
    \caption{International Sample : Exchange-Traded Funds' aggregate ownership share and underlying securities' liquidity}
    \label{tab:Liquidity:Intl:Comp}
  \begin{tabular}{lcccc}
    \toprule
    &   \textbf{Amihud}   & \textbf{Amihud w/inst. o'ship} &   \textbf{Bid-Ask}  & \textbf{Bid-Ask w/inst. o'ship}  \\
    \midrule
    \textbf{Dep. Variable}                     &   AmihudNumerator   &        AmihudNumerator         &   PctBidAskSpread   &         PctBidAskSpread          \\
    \textbf{Estimator}                         &       PanelOLS      &            PanelOLS            &       PanelOLS      &             PanelOLS             \\
    \textbf{No. Observations}                  &       1465561       &            1465555             &       1124265       &             1124260              \\
    \textbf{Cov. Est.}                         &    Driscoll-Kraay   &         Driscoll-Kraay         &    Driscoll-Kraay   &          Driscoll-Kraay          \\
    \textbf{R-squared}                         &        0.0285       &             0.0285             &        0.0227       &              0.0227              \\
    \textbf{R-Squared (Within)}                &        0.0287       &             0.0287             &        0.0375       &              0.0375              \\
    \textbf{R-Squared (Between)}               &       -0.3627       &            -0.3626             &        0.1785       &              0.1785              \\
    \textbf{R-Squared (Overall)}               &       -0.0089       &            -0.0089             &        0.0956       &              0.0956              \\
    \textbf{F-statistic}                       &      1.064e+04      &             8514.8             &        6433.1       &              4289.0              \\
    \textbf{P-value (F-stat)}                  &        0.0000       &             0.0000             &        0.0000       &              0.0000              \\
    \midrule
    \textbf{Intercept}                         &        0.0776       &             0.0776             &        0.1446       &              0.1446              \\
    \textbf{ }                                 &       (12.831)      &            (12.829)            &       (25.012)      &             (25.021)             \\
    \textbf{PctSharesHeldETF\_1lag}            &        0.0038       &             0.0038             &        0.0613       &              0.0610              \\
    \textbf{ }                                 &       (0.5715)      &            (0.5817)            &       (4.9367)      &             (4.9518)             \\
    \textbf{np.log(CompanyMarketCap\_1lag)}    &       -0.0058       &            -0.0058             &       -0.0061       &             -0.0061              \\
    \textbf{ }                                 &      (-21.501)      &           (-21.502)            &      (-23.176)      &            (-23.183)             \\
    \textbf{BookToMarketRatio\_1lag}           &        0.0012       &             0.0012             &      1.107e-05      &            1.108e-05             \\
    \textbf{ }                                 &       (1.0328)      &            (1.0328)            &       (0.6876)      &             (0.6887)             \\
    \textbf{np.log(AmihudDenominator)}         &        0.0042       &             0.0042             &                     &                                  \\
    \textbf{ }                                 &       (16.998)      &            (16.998)            &                     &                                  \\
    \textbf{PctSharesHeldOtherMutual\_1lag}    &                     &           -1.341e-05           &                     &            -1.92e-06             \\
    \textbf{ }                                 &                     &           (-0.9416)            &                     &            (-0.2806)             \\
    \textbf{Volatility\_1lag}                  &                     &                                &        0.1772       &              0.1772              \\
    \textbf{ }                                 &                     &                                &       (13.426)      &             (13.426)             \\
    \textbf{PctSharesHeldPension\_1lag}        &                     &                                &                     &              0.0314              \\
    \textbf{ }                                 &                     &                                &                     &             (1.1500)             \\
    \midrule
    \textbf{Effects}                           &        Entity       &             Entity             &        Entity       &              Entity              \\
    & Time & Time & Time & Time\\
    \bottomrule
  \end{tabular}
  %\caption{Model Comparison}
\end{center}
T-stats reported in parentheses.

    \end{table}
  }
\end{landscape}
\subsection{ETF ownership and concerns about pricing efficiency}
\label{sec:Results:sub:Efficiency}
\subsubsection{U.S. sample}
\autoref{tab:Efficiency:US:Comp} gathers panel OLS regressions at quarterly frequency for the U.S. stocks sample. The row of ETF ownership's coefficients (and associated t-stats below) attempting to explain either the variance ratio or the absolute deviation to 1 of the variance ratio is not statistically significant, although the displayed signs (mean reversion and thus deviation from random walk) seem to go in the expected direction. This comes generally in contradiction with \textcite{Ben-David2018}'s findings and running the same (untabulated) analysis on subsamples sorted on market capitalization quartiles does not yield a different result. As for the estimation of ETF ownership and volatility, the estimation strategy in the original paper relies on an instrumental variable, a binary depending on the switch of the security from an index to another (the Russell 1000 and 2000 indices) in order to avoid endogeneity between volatility and ETF ownership. Unlike the aforementioned analysis, we cannot say in the light of results for the U.S. stocks sample that ownership through exchange-traded funds yields more mean reversion over non-overlapping 5-day windows. The share of the variance ratios' heterogeneity explained within entities is extremely low.

\begin{landscape}
  {\linespread{1.0}
    \begin{table}[htbp]
      \begin{center}
  \caption{U.S. Sample : Exchange-Traded Funds' aggregate ownership share and weekly mean reversion of underlying securities}
  \label{tab:Efficiency:US:Comp}
  \begin{tabular}{lcccc}
\toprule
                                           &  \textbf{Abs. VR}  & \textbf{Abs. VR w/inst. o'ship} &    \textbf{VR}     & \textbf{VR w/inst. o'ship}  \\
\midrule
\textbf{Dep. Variable}                     &       absVR        &              absVR              &         VR         &             VR              \\
\textbf{Estimator}                         &      PanelOLS      &             PanelOLS            &      PanelOLS      &          PanelOLS           \\
\textbf{No. Observations}                  &       126851       &              126847             &       126851       &           126847            \\
\textbf{Cov. Est.}                         &   Driscoll-Kraay   &          Driscoll-Kraay         &   Driscoll-Kraay   &       Driscoll-Kraay        \\
\textbf{R-squared}                         &       0.0079       &              0.0081             &       0.0040       &           0.0041            \\
\textbf{R-Squared (Within)}                &       0.0106       &              0.0106             &       0.0040       &           0.0042            \\
\textbf{R-Squared (Between)}               &       0.2814       &              0.2820             &       0.1376       &           0.1385            \\
\textbf{R-Squared (Overall)}               &       0.0318       &              0.0320             &       0.0163       &           0.0165            \\
\textbf{F-statistic}                       &       123.79       &              91.734             &       61.843       &           46.323            \\
\textbf{P-value (F-stat)}                  &       0.0000       &              0.0000             &       0.0000       &           0.0000            \\
\midrule
\textbf{Intercept}                         &       0.8523       &              0.8532             &       0.8735       &           0.8579            \\
\textbf{ }                                 &      (12.280)      &             (12.208)            &      (7.6512)      &          (7.4328)           \\
\textbf{PctSharesHeldETF\_1lag}            &       0.0096       &              0.0218             &      -0.1205       &          -0.1005            \\
\textbf{ }                                 &      (0.1484)      &             (0.3373)            &     (-1.4081)      &         (-1.1660)           \\
\textbf{np.log(CompanyMarketCap\_1lag)}    &      -0.0248       &             -0.0248             &       0.0044       &           0.0052            \\
\textbf{ }                                 &     (-7.5202)      &            (-7.4640)            &      (0.8038)      &          (0.9458)           \\
\textbf{InvClose\_1lag}                    &       0.1276       &              0.1277             &      -0.2002       &          -0.1986            \\
\textbf{ }                                 &      (5.5223)      &             (5.5778)            &     (-6.0619)      &         (-5.9225)           \\
\textbf{AmihudRatio\_1lag}                 &       9.9267       &              9.9488             &      -15.135       &          -15.117            \\
\textbf{ }                                 &      (3.1431)      &             (3.1455)            &     (-3.0288)      &         (-3.0279)           \\
\textbf{PctBidAskSpread\_1lag}             &       0.9314       &              0.9286             &      -1.2190       &          -1.2179            \\
\textbf{ }                                 &      (5.5878)      &             (5.6027)            &     (-3.5732)      &         (-3.5633)           \\
\textbf{BookToMarketRatio\_1lag}           &     -3.498e-06     &            4.433e-05            &     -1.257e-05     &         6.777e-05           \\
\textbf{ }                                 &     (-0.5195)      &             (3.9695)            &     (-1.4303)      &          (3.2826)           \\
\textbf{RetPast12to7M\_1lag}               &       0.0003       &            -9.153e-05           &      -0.0006       &          -0.0013            \\
\textbf{ }                                 &      (0.5981)      &            (-0.2511)            &     (-0.9013)      &         (-2.2311)           \\
\textbf{GrossProfitability\_1lag}          &       0.0011       &              0.0012             &       0.0064       &           0.0062            \\
\textbf{ }                                 &      (0.2252)      &             (0.2454)            &      (1.2767)      &          (1.2635)           \\
\textbf{PctSharesHeldOtherMutual\_1lag}    &                    &             -0.0053             &                    &          -0.0086            \\
\textbf{ }                                 &                    &            (-3.9260)            &                    &         (-3.6908)           \\
\textbf{PctSharesHeldPension\_1lag}        &                    &              11.292             &                    &          -3.7029            \\
\textbf{ }                                 &                    &             (4.1468)            &                    &         (-0.5362)           \\
\textbf{PctSharesHeldHedge\_1lag}          &                    &             -0.1294             &                    &          -0.1145            \\
\textbf{ }                                 &                    &            (-0.8927)            &                    &         (-0.6268)           \\
\midrule
\textbf{Effects}                           &       Entity       &              Entity             &       Entity       &           Entity            \\
& Time & Time & Time & Time\\
\bottomrule
\end{tabular}
\end{center}
T-stats reported in parentheses.

    \end{table}
  }
\end{landscape}

\subsubsection{International sample}
If no evidence of an effect of ETF ownership over the monthly volatility of daily could be shown, \autoref{tab:Efficiency:Intl:Comp} brings solid arguments (at $\alpha = 1\%$) for ETFs still playing a role in the short run, more precisely increasing weekly mean reversion. Again, similar to the U.S. results summary, the coefficient of determination is nearly 0 according to the within specification and relatively higher across entities: up to 10\% for absolute variance ratio deviation. The closest analysis found, \textcite{Ben-David2018}, is an instrumental-variable two-stage estimation for U.S. stocks and no information is available regarding the coefficient of determination.

\bigskip
Despite the, at best fragile and heterogenuous, evidence exhibited in this subsection, more detail is reported in the appendix, \autoref{app:sec:DetailedResults:Efficiency}, p.\pageref{app:sec:DetailedResults:Efficiency}, about the panel estimation results testing the relationship between the share of ETF ownership in securities and the statistics of mean reversion.

\begin{landscape}
  {\linespread{1.0}
    \begin{table}[htbp]
      \begin{center}
  \caption{International Sample : Exchange-Traded Funds' aggregate ownership share and weekly mean reversion of underlying securities}
  \label{tab:Efficiency:Intl:Comp}
  \begin{tabular}{lcccc}
    \toprule
    &  \textbf{Abs. VR}  & \textbf{Abs. VR w/inst. o'ship} &    \textbf{VR}     & \textbf{VR w/inst. o'ship}  \\
    \midrule
    \textbf{Dep. Variable}                     &       absVR        &              absVR              &         VR         &             VR              \\
    \textbf{Estimator}                         &      PanelOLS      &             PanelOLS            &      PanelOLS      &          PanelOLS           \\
    \textbf{No. Observations}                  &       592054       &              592054             &       591848       &           591848            \\
    \textbf{Cov. Est.}                         &   Driscoll-Kraay   &          Driscoll-Kraay         &   Driscoll-Kraay   &       Driscoll-Kraay        \\
    \textbf{R-squared}                         &       0.0049       &              0.0049             &       0.0014       &           0.0014            \\
    \textbf{R-Squared (Within)}                &       0.0055       &              0.0055             &       0.0017       &           0.0017            \\
    \textbf{R-Squared (Between)}               &       0.0991       &              0.0992             &       0.0757       &           0.0758            \\
    \textbf{R-Squared (Overall)}               &       0.0099       &              0.0099             &       0.0104       &           0.0104            \\
    \textbf{F-statistic}                       &       354.08       &              314.75             &       98.569       &           87.752            \\
    \textbf{P-value (F-stat)}                  &       0.0000       &              0.0000             &       0.0000       &           0.0000            \\
\midrule
    \textbf{Intercept}                         &       0.9251       &              0.9250             &       0.5297       &           0.5304            \\
    \textbf{ }                                 &      (22.527)      &             (22.576)            &      (6.1864)      &          (6.1821)           \\
    \textbf{PctSharesHeldETF\_1lag}            &       0.2529       &              0.2539             &      -0.6122       &          -0.6199            \\
    \textbf{ }                                 &      (2.7216)      &             (2.6854)            &     (-3.6541)      &         (-3.6808)           \\
    \textbf{np.log(CompanyMarketCap\_1lag)}    &      -0.0248       &             -0.0248             &       0.0106       &           0.0106            \\
    \textbf{ }                                 &     (-13.525)      &            (-13.558)            &      (2.7762)      &          (2.7598)           \\
    \textbf{InvClose\_1lag}                    &       0.0015       &              0.0015             &      -0.0146       &          -0.0146            \\
    \textbf{ }                                 &      (0.1169)      &             (0.1168)            &     (-0.6261)      &         (-0.6257)           \\
    \textbf{AmihudRatio\_1lag}                 &       0.2138       &              0.2138             &      -0.2476       &          -0.2475            \\
    \textbf{ }                                 &      (3.3927)      &             (3.3923)            &     (-4.5128)      &         (-4.5122)           \\
    \textbf{PctBidAskSpread\_1lag}             &       0.2117       &              0.2117             &      -0.2938       &          -0.2937            \\
    \textbf{ }                                 &      (5.8326)      &             (5.8319)            &     (-5.4872)      &         (-5.4834)           \\
    \textbf{BookToMarketRatio\_1lag}           &       0.0001       &              0.0001             &      -0.0001       &          -0.0001            \\
    \textbf{ }                                 &      (1.6458)      &             (1.6464)            &     (-0.5935)      &         (-0.5995)           \\
    \textbf{RetPast12to7M\_1lag}               &     -6.928e-05     &            -6.928e-05           &     -7.459e-05     &         -7.461e-05          \\
    \textbf{ }                                 &     (-2.6264)      &            (-2.6265)            &     (-1.7080)      &         (-1.7081)           \\
    \textbf{GrossProfitability\_1lag}          &      -0.0041       &             -0.0041             &       0.0036       &           0.0036            \\
    \textbf{ }                                 &     (-3.2319)      &            (-3.2325)            &      (1.5214)      &          (1.5183)           \\
    \textbf{PctSharesHeldOtherMutual\_1lag}    &                    &             -0.0004             &                    &           0.0026            \\
    \textbf{ }                                 &                    &            (-0.2369)            &                    &          (0.7347)           \\
\midrule
    \textbf{Effects}                           &       Entity       &              Entity             &       Entity       &           Entity            \\
    & Time & Time & Time & Time\\
    \bottomrule
  \end{tabular}
  %\caption{Model Comparison}
\end{center}
T-stats reported in parentheses.

    \end{table}
  }
\end{landscape}
