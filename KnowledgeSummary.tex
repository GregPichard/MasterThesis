\section{State of academic knowledge}
The advent of Exchange-Traded Funds (hereafter ETF) takes place in an ongoing debate that have led modern financial economics at least since Fama's paper about the Efficient Market Hypothesis (hereafter EMH) around 1965. The 2014 Nobel Prize granted to Eugene Fama \textit{and} Robert Schiller, EMH's main critic\footnote{As a matter of completeness, let us not forget the third recipient, Lars Peter Hansen, who developped the econometric methods used to test the EMH} has obviously acknowledged the secular importance of the debate but the discussion has certainly not been closed since then.
- Masses have accessed the financial markets during the post-war
- Institutionalization of investment
- The rise of the investment company
- Mutual funds have claimed to provide a performance relative to benchmarks that are reference indices, for example the market-capitalization-weight stock index.
- On the other hand, partly as a consequence, passive investing has become more easily available for small, individual investors thanks to a new type open-end mutual fund : the index fund.
- ETFs differ from \textit{nowadays called} traditional index funds in that they offer permanent liquidity throughout trading hours, they publish the fund's Net Asset Value at regular intervals and the creation/redemption of shares is the task of institutions entitled to exchange a certain in kind securities for a newly created batch of ETF shares or to do the opposite trade with the fund sponsor to redeem ETF shares. The close tracking of the index is made possible by the privilege that authorized participants, a designation essentially applied to market makers, have to perform an arbitrage between the NAV and the actual value of underlying securities on the exchange. 

\parencite{Ben-David2011}
In a recent working paper called \textit{The Active World of Passive Investing}, \cite{Easley2018} claim that the shift from mutual funds and individual securities to ETFs is not simply a shift from active to passive investing, that distinction being ``antiquated''. Popular statistical metrics such as the Active Share -- which measures the absolute deviations of a portfolio holdings compared with a value-weighted benchmark -- and the Tracking Error -- i.e. theannualized standard deviation of the daily difference between the portfolio and market returns -- show that the decennial increase in ETFs share is actually fueled by active investing, including smart beta products whose activeness could be debated, because they essentially try to systematically capture risk premia justified by factors studied in the academic literature (most importantly, value and momentum factors). Nevertheless, a worldwide phenomenon has been at work leading to the mutual fund industry specializing towards passive investment and the ETF products, first designed to provide an efficient tracking of value-weighted stock indices, mutating to virtually any flavour of index, whether focusing on specific sizes, sectors, regions or factors.  
