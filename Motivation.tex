\section{Motivation}
\label{sec:Motivation}
\begin{center}
  \textsc{Remains to be developed : this is the proper introduction of the thesis, insisting on the reasons why the topic is relevant and a current issue in research : the rise of passive investing and alternative risk premia products availability for a broader segement of the market; recent notable evidence based on a track record of about 15 to 20 years of measurable ETF existence on the market; very important of volume on exchanges attributable to ETF raises the question of its impact; very visible benefits for ETF investors in terms of cost, intraday liquidity, less information about the disadvantages for the market as a whole; ETFs already linked with abrupt non-fundamental market movements, quickly reversed (flash crashes) : they are connected with concerns about high-frequency trading; fears expressed by some commentators if the market becomes more passive and the role of ETF continues to grow; research about earlier forms of index investing (e.g. futures) has yielded results that should be updated and compared.\\Detail the choice of variables of interest, documented evidence and how this paper extends existing research by 1. investigating three sub-topics into a single dataset and 2. expanding the coverage to international (i.e. non-U.S.) equity.}
  \end{center}

\subsection{Structure of the paper}
The research questions imply a tranmission mechanism between two layers of trading, thus an overview of both the structure and the context is needed. First, focusing on the specific nature of ETFs and on distinctive characteristics of the ETF markets, \autoref{sec:ETFCharacteristics} provides a brief institutional summary. \autoref{sec:Chronology} navigates along the time dimension in order to put the recent expansion in context. The review of literature in \autoref{sec:Literature} has a narrow scope since the focus of this part is exclusively on recent published research about the impact of ETFs on stocks. Next, the data used are described in \autoref{sec:Data}, before the more \autoref{sec:Methodology} explicits how the key variables are computed, what the best-known theoretical explanations for various controls are and which model specifications are estimated. This part draws heavily on multiple strands of the asset pricing literature. Logically, the outcome is disclosed and discussed in \autoref{sec:Results}. Finally, \autoref{sec:Conclusion} summarizes the main findings, draws conclusions regarding the research question, attempts both at showing how the current methodology could be extended and, admittedly a perilous exercise, tries discussing what the perspectives of ETF markets and their effects might be in the future. 
