\section{Motivation}
\label{sec:Motivation}
\subsection{Context}
Few investment vehicles have reached both a significant role in financial markets and a great variety of investors in such a short period, as exchange-traded funds have during the last three decades. Their staggering growth has been fueled by various factors that we will try to describe before suggesting aspects of the underlying securities that may be influenced by the growth of asset under management within ETFs. Index investing already existed before ETFs made their debut but the use of tracking instruments has drastically evolved, as shows the fact that ETFs are on average held for a much shorter time than stocks. However, there could be so far hidden disadvantages with the new place of ETFs, hidden because they do not only affect ETF investors but the equity market as a whole.

\subsection{Research questions}
By studying several \emph{unintended effects}, the aim is to show that the problems may exist on several aspects. It is of scientific but also regulatory interest to determine whether and to what extent ETF, a class of index-tracking products, cause the securities they own to become more volatile, more costly to trade and less efficient. Some authors have even designated those effects as the ``dark side of exchange-traded funds''\footcite{Israeli2017}. Therefore three simple questions giving its structure to this thesis are:
\begin{itemize}
\item Do ETFs make the securities they hold more volatile in the short run ?
\item Is there a change in the liquidity of underlying securities synthetic baskets are available for uninformed investors ?
  \item Is it possible, according to some claims expressed in the public, that the prices react to ETF trading in addition of fundamental news and therefore are less efficient ?
\end{itemize}
The following sections design an empirical panel estimation in order to answer the questions, with a significant support from the existing literature, which provides methodological guidance and reality checks regarding the results.

\subsection{Structure of the paper}
The research questions imply a tranmission mechanism between two layers of trading, thus an overview of both the structure and the context is needed. First, focusing on the specific nature of ETFs and on distinctive characteristics of the ETF markets, \autoref{sec:ETFCharacteristics} provides a brief institutional summary. \autoref{sec:Chronology} navigates along the time dimension in order to put the recent expansion into context. The review of literature in \autoref{sec:Literature} has a narrow scope since the focus of this part is exclusively on recent published research about the impact of ETFs on stocks. Next, the data used are described in \autoref{sec:Data}, before \autoref{sec:Methodology} explicits the methodology : how the key variables are computed, what the best-known theoretical explanations for various controls are and which model specifications are estimated. This part draws heavily on multiple strands of the asset pricing literature. Logically, the outcome is disclosed and discussed in \autoref{sec:Results}. Finally, \autoref{sec:Conclusion} summarizes the main findings, draws conclusions regarding the research question and attempts at showing how the current methodology could be extended.

Three parts in the appendix provide additional information and systematic results that would not match the goal of concision in the main body: section~\autoref{app:sec:InstitutionalBackground} is an extension of the characteristics of ETFs; regarding econometrics, section~\autoref{app:sec:EconometricDefinitions} clarifies two aspects mentioned in estimation results summaries; section~\autoref{app:sec:DetailedResults} contains the exhaustive estimation results summarized in the main body.
