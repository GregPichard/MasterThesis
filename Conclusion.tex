\section{Conclusion and further research questions}
\label{sec:Conclusion}
\subsection{Summary of objectives and results}
The goal of this paper was to exhibit and explain the unintended effects of exchange-traded funds investing in stocks worldwide. Amounting to more close to 15\% of the U.S. overall market capitalization and one third of the volume exchanged, these structures are suspected to make the markets more volatile due to the arbitrage mechanism allowing them to be continuously priced close to their NAV. Especially, the volatility exhibited concerns temporary shocks driven by non-fundamental demand, before prices revert towards their initial value. Effects on liquidity have also been studied and opposite conclusions drawn regarding the actual effect.

The reconstitution of ETF holdings is the key component in this study since data are available at the granular fund-stock-month level over the period 1999-2018. Based on the extensive dataset of mutual fund holdings, which include ETFs as well as other distinct types of collective investment schemes, the aggregate evolution of ETF holdings for each referred company has been structured as a panel in which the \textit{individuals} populating the panel are the listed companies, in the U.S. and outside. The same procedure has been applied, as far as the category of fund was available, for the aforementioned funds. The quality of data is severely impaired by infrequent reporting of various categories of funds and whenever no update from the previous month is found, the fund holdings are assumed to be stale.

The replication part consists in dynamic panel OLS regressions of the volatility, liquidity and efficiency variables over the ETF aggregate holdings, relevant controls that are academically founded and whose construction is possible given the database. Those controls are, non exhaustively, related with the Fama-French-Carhart model, gross profitability and liquidity/cost of trading indicators that bear a relevance as volatility and liquidity determinants. The novel part, with regard to the known bibliography and besides treating several close research questions with the same dataset, is the expansion to the largest possible representative set of stocks and other approaching entities (e.g. American Depository Receipts) globally, over nearly two decades.

Despite a thorough concern in applying similar methodologies, the recent statements from the literature do not find a full confirmation throughout this paper. The volatility effect of (higher) ETF ownership in the U.S. is a statistically significant higher standard deviation of daily returns in the following month, \textit{ceteris paribus} but the extent of this difference is questionable : 0.8\% of a standard deviation (st. dev. over the sample : 2.6\%) for a standard deviation of ETF ownership (3.5\%). In comparison and using their best fits, \textcite{Ben-David2018} find coefficients of 7.7\% for S\&P 500 stocks and 5.3\% for Russell 3000 stocks between 2000 and 2015. The order of magnitude of this effect is between 5 and 10 lower in our analysis, based on the actual (and different) values of standard deviations employed. The standard deviation of ETF ownership is on average around 2\% in their sample whereas ours is closer to 3.5\%. In the international stocks' sample, the same effect is statistically equal to 0 ($\mathit{t-stat} = 0.98$).

Estimations of the liquidity effects of ETF ownership again are not uniform in terms of significance : only the average absolute return, i.e. the numerator in the \textcite{Amihud2002} ratio, increases when ETFs increase their share into U.S. stocks, whereas the bid-ask spread is statistically unchanged. The opposite stands for the foreign sample : only the bid-ask spread is affected to an economically narrow extent : for a 1 percentage point increase in ETF ownership, the spread is expected to rise by 6 basis points. As a comparison, \textcite{Israeli2017} link a similar increase by 1.7 bp in the next year's average bid-ask spread, using an alternative proxy for the trading cost.

Finally, the concern that ETF reduce the efficiency of their underlying stocks' prices has been studied and has been confirmed only outside the U.S.: indeed, the 5-day-to-1-day variance ratio is negatively correlated with ETF ownership. In other words, provided that the weekly volatility (from Monday to Monday) remains constant, on average the volatility of daily prices increases.

\subsection{Limitations and perspectives}
Several methodological aspects need to be mentioned since, only addressed as much as possible over the course of this thesis, they may limit the validity or the generality of results. More robustness checks would in all cases be necessary, in order to assess whether the significant results with regard to volatility do not concentrate in either subperiod and/or subgroup of stocks (whether grouped by country/region or industry). Some doubt is casted with regard to the availability of individual fund-stock-month holdings which serve as the basis to construct aggregate ETF holdings, as well as other aggregate fund categories' holdings. Accessing another source of data specialized into mutual fund (in the broad sense) common stock holdings, namely the S12 database maintained by Thomson Reuters and a subscription service on WRDS, would perhaps have yielded more accurate results, although it is unlikely that such a service is able to warrant any reporting delay.

Additionally, the causal relationship of ETF ownership on volatility through essentially passive investments tracking famous market-capitalization indices is not proven through the specifications tested. The direct estimation of the effect of ETF ownership on volatility may bear a subtle and essential caveat: there may exist an unknown cause, an omitted factor that would be correlated both with ETF ownership and volatility. A robust identification would alleviate this possible endogeneity bias by using an instrumental variable based on a restriction. For the instrument to be valid, it has to affect the volatility only through its correlation with ETF ownership. Since a significant share of ETFs track well-known indices, the literature (e.g. \textcite{Ben-David2018}) uses the inclusion/exclusion event of a given stock into the index to explain the variation of ETF ownership. The second step consists in an panel regression similar to the previous ones in which the observed ETF ownership is replaced with the fitted value from the first stage regression.

Reassuringly, a significant contribution studying the same question \parencite{Ben-David2018} has demonstrated the similar conclusions of instrumental variable and OLS models up to their magnitude, for more than 3500 U.S. stocks. Such a correction, already implemented in the literature, is a privileged avenue for further research especially in the international sample, provided that a source of those index holdings (e.g. MSCI World Index, Russell 3000 for the U.S.) is openly available, which was not the case at the time of this study.

Overall, despite its mentioned limitations, the intent of this paper was to bring a novel contribution on the increasingly studied and debated subjects of Exchange-Traded Fund's global influence over major equity markets. The novel aspects are especially, first, the construction of panel of more than four thousand U.S. stocks and more than sixteen thousand stocks on twenty-four other national stock markets over twenty years, and second the analysis of several important dimensions of risk -- volatility, liquidity and pricing efficiency -- upon which alarming claims have been made in the media and for which investors should be eager to find solid empirical evidence.
