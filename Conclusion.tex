\section{Conclusion and further research questions}
\label{sec:Conclusion}
\subsection{Summary of objectives and results}
\subsection{Limitations and perspectives}
Several methodological aspects need to be mentioned since, only addressed as much as possible over the course of this thesis, they may limit the validity or the generality of results. More robustness checks would in all cases be necessary, in order to assess whether the significant results with regard to volatility do not concentrate in either subperiod and/or subgroup of stocks (whether grouped by country/region or industry). Some doubt is casted with regard to the availability of individual fund-stock-month holdings which serve as the basis to construct aggregate ETF holdings, as well as other aggregate fund categories' holdings. Accessing another source of data specialized into mutual fund (in the broad sense) common stock holdings, namely the S12 database maintained by Thomson Reuters and a subscription service on WRDS, would perhaps have yielded more accurate results, although it is unlikely that such a service is able to warrant any reporting delay.

Additionally, the causal relationship of ETF ownership on volatility through essentially passive investments tracking famous market-capitalization indices is not proven through the specifications tested. The direct estimation of the effect of ETF ownership on volatility may bear a subtle and essential caveat: there may exist an unknown cause, an omitted factor that would be correlated both with ETF ownership and volatility. A robust identification would alleviate this possible endogeneity bias by using an instrumental variable based on a restriction. For the instrument to be valid, it has to affect the volatility only through its correlation with ETF ownership. Since a significant share of ETFs track well-known indices, the literature (e.g. \textcite{Ben-David2018}) uses the inclusion/exclusion event of a given stock into the index to explain the variation of ETF ownership. The second step consists in an panel regression similar to the previous ones in which the observed ETF ownership is replaced with the fitted value from the first stage regression.

Reassuringly, a significant contribution studying the same question \parencite{Ben-David2018} has demonstrated the similar conclusions of instrumental variable and OLS models up to their magnitude, for more than 3500 U.S. stocks. Such a correction, already implemented in the literature, is a privileged avenue for further research especially in the international sample, provided that a source of those index holdings (e.g. MSCI World Index, Russell 3000 for the U.S.) is openly available, which was not the case at the time of this study.

Overall, despite its mentioned limitations, the intent of this paper was to bring a novel contribution on the increasingly studied and debated subjects of Exchange-Traded Fund's global influence over major equity markets. The novel aspects are especially, first, the construction of panel of more than four thousand U.S. stocks and more than sixteen thousand stocks on twenty-four other national stock markets over twenty years, and second the analysis of several important dimensions of risk -- volatility, liquidity and pricing efficiency -- upon which alarming claims have been made in the media and for which investors should be eager to find solid empirical evidence.
