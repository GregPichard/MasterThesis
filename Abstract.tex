\begin{abstract}
  Over the first two decades of this century, exchange-traded funds have become a new standard both for institutional investors seeking intraday liquidity, low tracking error, and for individuals in search of a diversified, systematic and extremely unexpensive fund. Their design enables them to track benchmarks, e.g. equity indices, through arbitrage and the creation/removal of units depending on demand. The effects of this activity on underlying stocks are not well-known yet and ETFs have been accused to exacerbate volatility and make security prices less reflective of fundamental factors. The empirical analysis is run similarly over two subsamples as a replication and expansion of existing findings : U.S. stocks, which account for the location of most ETF investments, and a set of 24 developped and emerging national stock markets. The estimation is run at a monthly frequency except for variance ratios, measured quarterly and robust controls are included in line with the asset pricing literature, institutional ownership through other instruments, and time-invariant characteristics of companies. Results depend on the subsample but are less clear than suggested in the literature: in the United States, stocks' volatility tends to rise during the month following an increase in the share of ETF ownership relative to the market value, although to a very limited extent. The price impact of trades rises on average in the month following this investment but mean reversion does not become stronger at the weekly scale. In the international sample, no statistically significant evidence of a positive link between ETF ownership and volatility can be found, but efficiency over five days decreases and the bid-ask spread also becomes wider in conjunction with ETF interest. Overall, some bold statements share in the public regarding ETFs and passive investment making markets inefficient  do not tend to have realized yet but the arbitrage channel underpinning ETFs pricing may have introduced a new layer of liquidity risk into underlying stocks.     
\end{abstract}
