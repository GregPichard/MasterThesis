\section{Institutional background}
\label{app:sec:InstitutionalBackground}
As a matter of conciseness, the description of the legal framework in which ETFs operate and the institutional design (market participants) in \autoref{sec:ETFCharacteristics} (p.\pageref{sec:ETFCharacteristics}) has been kept at its shortest in the main body of this paper. This appendix aims at providing a more exhaustive overview of these topics.

\subsection{Roles}
In the main body, two main actors have been discussed : the ETF sponsor and authorized participants. Besides the two main actors and the clearing corporation, the proper daily trading and liquidity is made possible through :
\begin{description}
\item[custodians] do not only hold the assets of the ETF on behalf of the fund issuer : they reconcile positions with the index rules, in case the product replicates an index from a third party (e.g. the S\&P 500 index computed by \textit{Standard \& Poor's Dow Jones Indices LLC}), and more generally with ETF rules, which are made publicly available in the prospectus. Custodians, often banks, execute periodic rebalancings and adjust the portfolio when a corporate action (e.g. stock split, delisting, merger or acquisition) so demands.
\item[market makers] create liquidity in the secondary market by steadily providing quotes based on their calculation of the ETF's fair value. It is reported that a limited number of market markers trade each ETF, unlike (liquid) stocks for which a best bid and offer is computed out of various quotes. Nevertheless, there are exists a correlation between the volume traded, the spread on the ETF market and the number of market makers. 
\end{description}


\subsection{Legal and regulatory framework}
ETFs may have at least two legal forms, based on information relevant in the United States : they may be registered as regulated open-end investment companies -- exactly the same as mutual funds, intuitively -- or as unit investment trusts; for instance, the much-traded SPDR S\&P 500 ETF (mnemonic code SPY), nowadays sponsored and administered through State Street Global Advisors, is a unit investment trust. This legal structure prevents the funds adopting it from engaging in securities lending, which is one of the main sources of revenues of ETF sponsors. The latter legal form, which does not, in principle, exclude active investment objectives, offers the redemption of shares at Net Asset Value to the owner, which will sometimes be called the primary market as opposed to the secondary market which enables buyers and sellers to trade shares independent from the trust advisor. Open-end funds and unit investment trusts are the two main legal structures used by ETFs focusing on stocks, which is a subset of the whole (expanding) ETF universe.

In terms of management, unit investment trusts differ from open-end investment companies since they do not have a board of directors; since the portfolio is assumed fixed and is especially straightforward if such an ETF engages in systematic physical replication, a board of directors would in essence seem superfluous. Even more anecdotally, unit investment trusts are required by law to have a fixed end date, however far away it is: for instance, the ultimate termination date of the SPY ETF is scheduled 125 years after its initial deposit, i.e. on January 22, 2118.

Since the majority of ETF, both in entities and in value terms, is registered and managed in the United States of America, more emphasis has been put on the structures allowed and regulated under the Investment Company Act of 1940. More generally, the legal form chosen by ETFs differs in every regulatory system\footnote{European countries even seem to have at least kept their national regulatory frame, at least according to the names advertised by those structures. Harmonization nevertheless has already started through the so-called \textit{Undertakings for the Collective Investment of Transferable Securities}, better known as the UCITS acronym which only labels compliant funds and not debt products such as notes, certificates, etc. Reportedly, as of 2017, three quarters of EU ordinary ETF investorshold UCITS funds. (Source : \url{https://www.justeft.com/uk/news/etf/legal-structure-of-etfs-ucits.html} [Consulted April 24, 2019])}, which unfortunately makes an exhaustive comparison of all eligible legal structures untractable for this study.
