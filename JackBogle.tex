\section{Foreword}
\label{Foreword}
It is a moving event that the actual beginning of the research regarding this paper nearly coincides with the death of one of the fathers, if not the demiurge of that same research topic : Mr. John C.,``Jack'', Bogle, who was born in 1929 and passed out on January 17th, 2019, best known as the founder of \textit{The Vanguard Group} in 1975. He was already considered a legend in the investment world decades before his death and was duly praised by finance leaders and scholars who were far from sharing his views. It is interesting to notice that he was almost as famous as one of his contemporaries, namely Warren Buffett (born in 1930), chairman of conglomerate and investment company \textit{Berkshire Hathaway}, for achieving success on a fundamentally opposite view of his role as an investor. Mr. Buffett has built an empire with active bets on companies that he considered, at the time he purchased them, to be undervalued and with a high book-to-market ratio, hence showing his unique ability to exploit what can be modelled as a small set of alternative risk premia, in activities which he has a deep understanding of. Recent research about his multi-decade long winning streak has shown his success is more systematic than genius stock-picking and relies on drivers that the academic literature considers systematic with low correlation to the market premium. Still, the conclusion is often that he is an overwhelmingly successful disciple of strict principles written in the 1930s by Graham and Dodd at the Columbia Business School. Since then, the research has caught up and shed light on the value risk factor for instance, which can explain a significant share (while not all) of Buffett's financial performance over the last 50 years. Mr. Buffett is the living proof that the investor can generate long term risk-adjusted performance above the market return thanks to a mix of information available to investors and a selection based on fundamental valuation. 

On the other side, British weekly newspaper \textit{The Economist}, has deemed Jack Bogle's curriculum as the founder of a business that
\begin{quotation}
has radically changed money management by being boring and cheap.
\end{quotation}

The worry that Bogle's ultimate success with widespread adoption of index funds leads to less liquidity and less research, has been expressed by Sam Zell, the billionaire chairman and founder of \textit{Equity Group Investments}, a private investment firm. In an interview on \textit{Bloomberg Television} in which, among various topics he was asked to react, was remembering Jack Bogle. He says that
\begin{quotation}
Bogle's concept and what he advocated was terrific as long as it did not get to be too big a percentage, and I think we are at that point now where there is significant risk.
\end{quotation}
Obviously, as any market actor and company representative or owner that will answer Bloomberg journalists' question, Sam Zell may have interests that do not match those of Vanguard's, not necessarily as a direct competitor. That issue is also raised by the hosts and Sam Zell answers that his group of firms' business is not ``yet'' strongly impacted by the rise of passive investing, but that
\begin{quotation}
everybody is pretty concerned about the percentage of ownership of New York Stock Exchange companies, particularly the REITs\footnote{Real estate investment trusts own, manage and even in some cases finance real estate assets, for example commercial surfaces (offices), residential buildings as well as malls and hotels. Sam Zell's companies are especially active in real estate and another part of the interview to \textit{Bloomberg} was about comments (strong words as he seems to be famous for that form of communication) regarding the trends in commercial real estate.} owned by passive players. [\ldots] If your percentage [in index funds] gets too high, then you institutionalize mediocrity.
\end{quotation}
Later he admits the long-run rise of passive investing may even cause a positive effect on capital flows to private equity, at the expense of active managers in public companies (concern with regard to hedge funds), so the overall effect on Zell's businesses is probably only clear to himself, as a manager of a private company.

Despite having developped the first index fund at time the idea was only nascent\footnote{As Jack Bogle himself wrote in a 2012 piece of the \textit{Journal of Indexes} about his impact in the industry, the idea was probably not only his, but he was the first to achieve such a development, although as it was commonly viewed as a mere failure at that time. To quote his introduction, ``ideas are a dime; implementation is everything'' Among earlier attempts starting in the late 1960s, not always based on the S\&P 500 Index, he recalls about modelling by \textit{Wells Fargo Bank} for the pension fund of \textit{Samsonite}, then by a mutual fund firm from\dots Boston (of course, as his own Master's thesis from 1951 was inspired by a \textit{Fortune} magazine piece titled ``Big Money in Boston''), the \textit{American National Bank} in Chicago (later absorbed by \textit{J.P. Morgan Chase}), \textit{American Express} before its management stopped the project and finally an insurance and pension company for teachers advised by Milton Friedman. Mr. Bogle concludes : ``In all of these forays of indexing: ideas A+; implementation F.''}, Jack Bogle expressed harsh criticism of index-tracking ETFs, altough they present some tax advantages over open-end index funds. The so-called issues raised by these widespead and quick growth of their volume related to the use made by some investors rather than the very nature of ETF. Opposite to the idealized buy-and-hold (and most importantly but less often stated, ``rebalance'') passive portfolio, ETFs exhibit turnover rates in line with intraday trading. With the individual investor in mind during his whole career, which he served by cutting management fees and the fund \textit{load} of his products, Jack Bogle noticed that they were tools enabling speculation. An abundant literature he authored condemned speculation as a net financial loss for fund investors and ultimately a serious lack of stewardship by the very industry he had been part of for more than six decades. While it is possible to track similar indices with traditional index funds and ETFs, the actual use of those securities differ to a great extent : while the former, in line with Bogle's, are generally long-term buy-and-hold investments, the latter, continuously tradeable, may serve for hedging, short-selling and thus for arbitrage purpose.

In this paper, we are especially curious to find, if not why, at least how this improved trading characteristic of ETFs may have changed the underlying securities' markets. Admittedly, studying an issue discussed by few people outside Jack Bogle himself means standing on the shoulder of a giant.

This recount of ongoing debate that flourish in the financial press only serves as an element of context. Additionally, it would have been a historical faux-pas not to mention Jack Bogle's impact in the passive investing world. Yet the fact that the relative size of index funds holdings in general and ETF holdings of stocks in particular is discussed outside scientific publications tends to show that the question deserves to be asked and is the rationale for further enquiry. This paper humbly seeks to contribute to this major debate. 
