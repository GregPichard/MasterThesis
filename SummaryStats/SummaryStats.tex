This section aims first at providing an overview about the distribution of all variables in the upcoming regressions. In addition, although a thorough comparison with the emerging literature about ETF ownership effects on securities is beyond the scope of this paper, summary statistics help ensuring that the sample bears some resemblance with those used in other studies. Resemblance is obviously weaker, less precise and rigorous than similarity and one should for instance perform statistical tests such as the F-test or Bartlett's test for equal variances across populations, under normality assumptions, if two samples are available.

Last, summary statistics are also useful in order to filter outliers at both ends of distributions. The general policy for this matter is first to exclude data points that are obviously wrong because they are impossible : a typical situation of a ratio $\frac{\mathsf{\#ETF-owned\_shares}}{\mathsf{\#Shares\_outstanding}}$ above 100\% is excluded but some cases are critical because doubt is allowed : a delay in the reporting of fund holdings can yield ETF ownership ratios close to, and yet below 1. Such case, however unlikely, may as well happen and no assumption has been made regarding a maximum tolerated ETF ownership share because of a lack of relevant information. Then, for the sake of simplicity, variables are truncated rather than winsorised at the 99.99\% quantile; the choice of such a high threshold has been guided by the fact that, in some variables including the essential ETF ownership share, variance concentrates in the upper percentiles of the distribution. In other terms, this variable empirical distribution is right-skewed and the intuition is that higher percentiles are not necessarily irrelevant outliers, they may even constitute an important part of our sample. For instance, the international sample exhibits very low share of (U.S.-listed) ETF ownership: the median and the 75th percentiles are 0, the 90th is equal to 0.7\%, the 95th to 1.6\% and the 99th to 4.3\%. The value exceeded by only 0.01\% of the sample, and therefore considered the threshold for truncation, is 22.5\%. Although not disclosed in summary tables, ownership of other institutional investors are especially has an especially outside the U.S., which in turn raises two issues in regressions : first the usable panel is a small fraction of the overall collected sample and even for the usable intersection, the rank condition on the regressors matrix is sometimes violated unless two of those controls, the pension and hedge funds share of ownership, are excluded from the regression.

The summary statistics in \autoref{tab:SummaryTable:US} and \autoref{tab:SummaryTable:International}, for the U.S. and abroad respectively, are computed entire samples, hence before truncation; only unvalaible values, encoded as \texttt{NaN} (``not a number'') or as infinite, are excluded. The maxima are therefore absent from data used in following regressions, which seems a healthy precaution for ETF ownership as explained above, but also for several controls including notably the book-to-market ratio, the percent bid-ask spread and the previous 12-month return.

\autoref{subtab:SummaryTable:US:PanelB} and \autoref{subtab:SummaryTable:International:PanelB} are essentially used to control for low absolute linear correlation between regressors in order to avoid any risk of collinearity among them; such risk seems absent and no coefficient is above 0.5 in absolute value. Perhaps anecdotally, the ``naïve'' univariate correlation between ETF ownership and the volatility of daily stock returns is negative (approx. $-0.23$) in the U.S. and very close to zero outside the U.S. Similarly, the bid-ask spread, a measure of liquidity which is assumed inversely linked with liquidity, exhibits a negative $\rho$ with ETF ownership while we may expect, in our multivariate tests, a positive correlation.
{\linespread{1.0}
\begin{landscape}
\begin{table}[htbp]
\caption{U.S. Sample (monthly) : Summary Statistics}
\label{tab:SummaryTable:US}
\begin{subtable}[t]{\linewidth}
\subcaption{Summary statistics}
\label{subtab:SummaryTable:US:PanelA}
\begin{tabular}{lrrrrrrrr}
\toprule
{} & N (obs.) &    Mean & St. dev. &      Min. &    25\% &  Median &     75\% &       Max. \\
\midrule
Volatility                &   296405 &   0.026 &    0.020 &     0.001 &  0.015 &   0.021 &   0.031 &      1.618 \\
ETF Ownership             &   296405 &   0.028 &    0.035 &     0.000 &  0.001 &   0.016 &   0.043 &      0.990 \\
Book-to-market            &   296405 &   1.008 &  157.773 &  -196.776 &  0.243 &   0.440 &   0.742 &  79321.012 \\
Market cap. (\$ Mln.)      &   296405 & 699.951 & 2833.749 &     0.013 & 29.986 & 105.878 & 365.001 & 109943.606 \\
1/Price                   &   296405 &   0.089 &    0.120 &     0.000 &  0.026 &   0.048 &   0.097 &      1.000 \\
Rel. Bid-Ask spread       &   296405 &   0.005 &    0.014 &    -0.015 &  0.001 &   0.001 &   0.004 &      1.765 \\
Amihud ratio              &   296405 &   0.000 &    0.000 &     0.000 &  0.000 &   0.000 &   0.000 &      0.003 \\
Past 12-to-1-month return &   296405 &   0.155 &    0.654 &    -1.000 & -0.144 &   0.075 &   0.320 &     43.375 \\
Past 12-to-7-month return &   296405 &  -0.049 &    3.632 & -1868.259 & -0.120 &   0.037 &   0.161 &      0.965 \\
Gross profitability       &   296405 &   0.333 &    0.351 &    -0.850 &  0.149 &   0.272 &   0.439 &     36.028 \\
\bottomrule
\end{tabular}

\end{subtable}
%\newline
\medskip
\newline
\begin{subtable}[t]{\linewidth}
\subcaption{Pearson correlation coefficients}
\label{subtab:SummaryTable:US:PanelB}
\begin{tabular}{llrrrrrrrrrr}
\toprule
                    &      &    (1) &    (2) &    (3) &    (4) &    (5) &    (6) &    (7) &   (8) &   (9) &  (10) \\
\midrule
Volatility & (1) &  1.000 &        &        &        &        &        &        &       &       &       \\
ETF Ownership & (2) & -0.149 &  1.000 &        &        &        &        &        &       &       &       \\
Book-to-market & (3) &  0.005 & -0.003 &  1.000 &        &        &        &        &       &       &       \\
Market cap. (\$ Mln.) & (4) & -0.101 &  0.028 & -0.001 &  1.000 &        &        &        &       &       &       \\
1/Price & (5) &  0.286 & -0.217 &  0.004 & -0.126 &  1.000 &        &        &       &       &       \\
Rel. Bid-Ask spread & (6) &  0.251 & -0.186 &  0.000 & -0.072 &  0.312 &  1.000 &        &       &       &       \\
Amihud ratio & (7) &  0.085 & -0.025 & -0.001 & -0.009 &  0.083 &  0.114 &  1.000 &       &       &       \\
Past 12-to-1-month return & (8) & -0.052 & -0.001 & -0.005 &  0.002 & -0.089 & -0.065 & -0.018 & 1.000 &       &       \\
Past 12-to-7-month return & (9) & -0.035 &  0.003 & -0.005 &  0.005 & -0.026 & -0.013 & -0.004 & 0.010 & 1.000 &       \\
Gross profitability & (10) &  0.001 & -0.008 &  0.000 &  0.001 &  0.003 &  0.002 & -0.001 & 0.052 & 0.002 & 1.000 \\
\bottomrule
\end{tabular}

\end{subtable}
\end{table}
\clearpage
\begin{table}[htbp]
\caption{International Sample : Summary Statistics}
\label{tab:SummaryTable:International}
\begin{subtable}[t]{\linewidth}
\subcaption{Summary statistics}
\label{subtab:SummaryTable:International:PanelA}
\begin{tabular}{lrrrrrrrr}
\toprule
{} & N (obs.) &      Mean &   St. dev. &       Min. &     25\% &  Median &      75\% &         Max. \\
\midrule
Volatility                &  1338479 &     0.420 &      0.125 &      0.000 &   0.397 &   0.418 &    0.441 &      109.046 \\
ETF Ownership             &  1338479 &     0.003 &      0.012 &      0.000 &   0.000 &   0.000 &    0.001 &        0.950 \\
Book-to-market            &  1338479 &     0.875 &     33.362 & -30363.758 &   0.360 &   0.693 &    1.237 &     1102.476 \\
Market cap. (\$ Mln.)      &  1338479 & 14351.004 & 214744.406 &      0.001 & 113.098 & 609.315 & 3385.835 & 39864005.168 \\
1/Price                   &  1338479 &     0.082 &      0.551 &      0.000 &   0.001 &   0.004 &    0.060 &      333.333 \\
Rel. Bid-Ask spread       &  1338479 &     0.013 &      0.039 &     -2.000 &   0.002 &   0.005 &    0.013 &        2.000 \\
Amihud ratio              &  1338479 &     0.000 &      0.004 &     -0.000 &   0.000 &   0.000 &    0.000 &        2.450 \\
Past 12-to-1-month return &  1338479 &     0.162 &      1.381 &     -1.000 &  -0.159 &   0.048 &    0.317 &     1249.000 \\
Past 12-to-7-month return &  1338479 &    -0.036 &      6.559 &  -5282.798 &  -0.128 &   0.019 &    0.151 &        0.997 \\
Gross profitability       &  1338479 &     0.282 &      0.389 &    -79.835 &   0.121 &   0.215 &    0.365 &       76.409 \\
\bottomrule
\end{tabular}

\end{subtable}
\medskip
\newline
\begin{subtable}[t]{\linewidth}
\subcaption{Pearson correlation coefficients}
\label{subtab:SummaryTable:International:PanelB}
\begin{tabular}{llrrrrrrrrrr}
\toprule
                    &  & (1) & (2) &  (3) & (4) & (5) & (6) & (7) & (8) & (9) & (10)\\
\midrule
Volatility & (1) &  1.000 &        &        &        &        &        &        &       &        &       \\
ETF Ownership & (2) & -0.004 &  1.000 &        &        &        &        &        &       &        &       \\
Book-to-market & (3) & -0.001 & -0.002 &  1.000 &        &        &        &        &       &        &       \\
Market cap. (\$ Mln.) & (4) & -0.000 &  0.079 & -0.000 &  1.000 &        &        &        &       &        &       \\
1/Price & (5) &  0.000 & -0.029 & -0.002 & -0.011 &  1.000 &        &        &       &        &       \\
Rel. Bid-Ask spread & (6) &  0.018 & -0.073 & -0.002 & -0.018 &  0.175 &  1.000 &        &       &        &       \\
Amihud ratio & (7) &  0.000 & -0.001 & -0.000 & -0.000 &  0.021 &  0.044 &  1.000 &       &        &       \\
Past 12-to-1-month return & (8) & -0.000 & -0.003 &  0.000 &  0.001 & -0.004 &  0.003 & -0.000 & 1.000 &        &       \\
Past 12-to-7-month return & (9) &  0.000 &  0.000 &  0.000 &  0.000 &  0.000 & -0.011 & -0.000 & 0.001 &  1.000 &       \\
Gross profitability & (10) & -0.000 & -0.010 &  0.021 & -0.004 & -0.003 &  0.012 & -0.001 & 0.003 & -0.040 & 1.000 \\
\bottomrule
\end{tabular}

\end{subtable}
\end{table}
\end{landscape}
}
