This section aims first at providing an overview about the distribution of all variables in the upcoming regressions. In addition, although a thorough comparison with the emerging literature about ETF ownership effects on securities is beyond the scope of this paper, summary statistics help ensuring that the sample bears some resemblance with those used in other studies. Resemblance is obviously weaker, less precise and rigorous than similarity and one should for instance perform statistical tests such as the F-test or Bartlett's test for equal variances across populations, under normality assumptions, if two samples are available. Since only one sample is available in our case, such comparisons are actually impossible. Lastly, summary statistics are also useful in order to filter outliers at both ends of distributions.

{\linespread{1.0}
\begin{landscape}
\begin{table}[htbp]
\label{tab:SummaryTable:US}
\caption{U.S. Sample (monthly) : Summary Statistics}
\begin{subtable}[t]{\linewidth}
\label{subtab:SummaryTable:US:PanelA}
\subcaption{Summary statistics}
\begin{tabular}{lrrrrrrrr}
\toprule
{} & N (obs.) &    Mean & St. dev. &      Min. &    25\% &  Median &     75\% &       Max. \\
\midrule
Volatility                &   300181 &   0.027 &    0.026 &     0.000 &  0.015 &   0.021 &   0.032 &      1.618 \\
ETF Ownership             &   300181 &   0.028 &    0.035 &     0.000 &  0.001 &   0.015 &   0.042 &      0.990 \\
Book-to-market            &   300181 &   1.001 &  156.778 &  -196.776 &  0.243 &   0.439 &   0.740 &  79321.012 \\
Market cap. (\$ Mln.)      &   300181 & 716.934 & 2846.498 &     0.013 & 30.072 & 107.263 & 374.055 & 109943.606 \\
1/Price                   &   300181 &   0.089 &    0.120 &     0.000 &  0.026 &   0.048 &   0.098 &      1.000 \\
Rel. Bid-Ask spread       &   300181 &   0.005 &    0.014 &    -0.015 &  0.000 &   0.001 &   0.004 &      1.765 \\
Amihud ratio              &   300181 &   0.000 &    0.000 &     0.000 &  0.000 &   0.000 &   0.000 &      0.101 \\
Past 12-to-1-month return &   300181 &   0.154 &    0.652 &    -1.000 & -0.143 &   0.075 &   0.319 &     43.375 \\
Past 12-to-7-month return &   300181 &  -0.049 &    3.609 & -1868.259 & -0.119 &   0.037 &   0.161 &      0.965 \\
Gross profitability       &   300181 &   0.332 &    0.349 &    -0.850 &  0.150 &   0.272 &   0.438 &     36.028 \\
\bottomrule
\end{tabular}

\end{subtable}
%\newline
\medskip
\newline
\begin{subtable}[t]{\linewidth}
\label{subtab:SummaryTable:US:PanelB}
\subcaption{Pearson correlation coefficients}
\begin{tabular}{llrlllllllll}
\toprule
                    &      &    (1) &    (2) &    (3) &    (4) &    (5) &    (6) &    (7) &   (8) &   (9) &  (10) \\
\midrule
Volatility & (1) &  1.000 &        &        &        &        &        &        &       &       &       \\
ETF Ownership & (2) & -0.229 &  1.000 &        &        &        &        &        &       &       &       \\
Book-to-market & (3) & -0.000 & -0.003 &  1.000 &        &        &        &        &       &       &       \\
Market cap. (\$ Mln.) & (4) & -0.001 &  0.028 & -0.001 &  1.000 &        &        &        &       &       &       \\
1/Price & (5) &  0.010 & -0.217 &  0.004 & -0.126 &  1.000 &        &        &       &       &       \\
Rel. Bid-Ask spread & (6) &  0.437 & -0.186 &  0.000 & -0.072 &  0.312 &  1.000 &        &       &       &       \\
Amihud ratio & (7) &  0.016 & -0.001 & -0.000 & -0.001 &  0.005 &  0.032 &  1.000 &       &       &       \\
Past 12-to-1-month return & (8) & -0.044 & -0.001 & -0.005 &  0.002 & -0.089 & -0.065 & -0.002 & 1.000 &       &       \\
Past 12-to-7-month return & (9) & -0.006 &  0.003 & -0.005 &  0.005 & -0.026 & -0.013 & -0.000 & 0.010 & 1.000 &       \\
Gross profitability & (10) &  0.019 & -0.008 &  0.000 &  0.001 &  0.003 &  0.002 & -0.001 & 0.052 & 0.002 & 1.000 \\
\bottomrule
\end{tabular}

\end{subtable}
\end{table}
\clearpage
\begin{table}[htbp]
\label{tab:SummaryTable:International}
\caption{International Sample : Summary Statistics}
\begin{subtable}[t]{\linewidth}
\label{subtab:SummaryTable:International:PanelA}
\subcaption{Summary statistics}
\begin{tabular}{lrrrrrrrr}
\toprule
{} & N (obs.) &      Mean &   St. dev. &       Min. &     25\% &  Median &      75\% &         Max. \\
\midrule
Volatility                &  1338479 &     0.420 &      0.125 &      0.000 &   0.397 &   0.418 &    0.441 &      109.046 \\
ETF Ownership             &  1338479 &     0.003 &      0.012 &      0.000 &   0.000 &   0.000 &    0.001 &        0.950 \\
Book-to-market            &  1338479 &     0.875 &     33.362 & -30363.758 &   0.360 &   0.693 &    1.237 &     1102.476 \\
Market cap. (\$ Mln.)      &  1338479 & 14351.004 & 214744.406 &      0.001 & 113.098 & 609.315 & 3385.835 & 39864005.168 \\
1/Price                   &  1338479 &     0.082 &      0.551 &      0.000 &   0.001 &   0.004 &    0.060 &      333.333 \\
Rel. Bid-Ask spread       &  1338479 &     0.013 &      0.039 &     -2.000 &   0.002 &   0.005 &    0.013 &        2.000 \\
Amihud ratio              &  1338479 &     0.000 &      0.004 &     -0.000 &   0.000 &   0.000 &    0.000 &        2.450 \\
Past 12-to-1-month return &  1338479 &     0.162 &      1.381 &     -1.000 &  -0.159 &   0.048 &    0.317 &     1249.000 \\
Past 12-to-7-month return &  1338479 &    -0.036 &      6.559 &  -5282.798 &  -0.128 &   0.019 &    0.151 &        0.997 \\
Gross profitability       &  1338479 &     0.282 &      0.389 &    -79.835 &   0.121 &   0.215 &    0.365 &       76.409 \\
\bottomrule
\end{tabular}

\end{subtable}
\medskip
\newline
\begin{subtable}[t]{\linewidth}
\label{subtab:SummaryTable:International:PanelB}
\subcaption{Pearson correlation coefficients}
\begin{tabular}{llrlllllllll}
\toprule
                    &      &    (1) &    (2) &    (3) &    (4) &    (5) &    (6) &    (7) &   (8) &    (9) &  (10) \\
\midrule
Volatility & (1) &  1.000 &        &        &        &        &        &        &       &        &       \\
ETF Ownership & (2) & -0.083 &  1.000 &        &        &        &        &        &       &        &       \\
Book-to-market & (3) &  0.020 & -0.003 &  1.000 &        &        &        &        &       &        &       \\
Market cap. (\$ Mln.) & (4) & -0.013 &  0.085 & -0.000 &  1.000 &        &        &        &       &        &       \\
1/Price & (5) &  0.093 & -0.071 & -0.004 & -0.032 &  1.000 &        &        &       &        &       \\
Rel. Bid-Ask spread & (6) &  0.256 & -0.081 & -0.002 & -0.018 &  0.077 &  1.000 &        &       &        &       \\
Amihud ratio & (7) &  0.058 & -0.001 &  0.009 & -0.000 &  0.013 &  0.052 &  1.000 &       &        &       \\
Past 12-to-1-month return & (8) &  0.008 & -0.003 & -0.007 &  0.001 & -0.008 &  0.003 & -0.000 & 1.000 &        &       \\
Past 12-to-7-month return & (9) & -0.001 &  0.000 & -0.002 &  0.000 &  0.000 & -0.011 & -0.000 & 0.001 &  1.000 &       \\
Gross profitability & (10) & -0.007 & -0.013 &  0.026 & -0.004 & -0.019 &  0.012 & -0.000 & 0.003 & -0.040 & 1.000 \\
\bottomrule
\end{tabular}

\end{subtable}
\end{table}
\end{landscape}
}
